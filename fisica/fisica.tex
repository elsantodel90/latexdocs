\documentclass{article}

% El archivo está codificado utf8.

\usepackage[utf8]{inputenc}


\usepackage[spanish]{babel}
\usepackage{amssymb}
\usepackage{amsmath,amsfonts}
\usepackage{amsthm}
\usepackage{graphicx}
\usepackage{xspace}
\usepackage{lmodern}
\usepackage[per-mode=fraction]{siunitx}

\newlength\longest

\usepackage{titling}

\setlength{\textwidth}{6.5in}
\setlength{\oddsidemargin}{0in}
\setlength{\textheight}{8.5in}
\setlength{\topmargin}{0.5in}
\setlength{\headheight}{0in}
\setlength{\headsep}{0in}

\def\lg{\mathop{\mathrm {lg}}\nolimits}

\def\rev{$\mathbb{R} \mbox{-espacio vectorial } $}
\def\qev{$\mathbb{Q} \mbox{-espacio vectorial } $}

% Dimensiones
\def\Dimensions{\mathcal D}
\def\Quantities{\mathcal Q}
\def\masa{\mathsf M}
\def\longitud{\mathsf L}
\def\tiempo{\mathsf T}
\def\carga{\mathsf Q}
\def\temperatura{\mathsf \Theta}



\title{\huge Física}
\author{Agustín Santiago Gutiérrez}
\date{}

\makeindex

\newtheorem{teorema}{{\sc Teorema}}
\newtheorem{definicion}{{\sc Definición}}
\newtheorem{corolario}{{\sc Corolario}}
\newtheorem{lema}{{\sc Lema}}

%Ejemplos de macros
%\newcommand{\Left}{\textbf{Left}\xspace}
%\newcommand{\Right}{\textbf{Right}\xspace}
%\newcommand{\Jota}{\ensuremath{\mathcal{J}}\xspace}
%\newcommand{\Juego}[2]{\ensuremath{\left \{ #1 | #2 \right \}}\xspace}

\begin{document}

\maketitle

\pagebreak

{

\thispagestyle{empty}
\null\vfill

\settowidth\longest{\large\itshape escrito por un Licenciado en Ciencias de la Computación,}

\centering
\parbox{\longest}{%
  \raggedright{\large\itshape%
   Un libro de física,   \\ 
  escrito por un Licenciado en Ciencias de la Computación, \\
  con énfasis en la formalidad matemática. \\ 
  ¿Qué puede salir mal?\par\bigskip
  }   
}


\vfill\vfill

}

\pagebreak

\tableofcontents

\pagebreak

\section{Prefacio}

\subsection{Objetivo del libro}

Este libro lo escribo para mí. La intención es recopilar lo que aprendo y estudio en forma de libro, en un estilo particular (muy 
formal y preciso, probablemente demasiado)
relativamente inusual en la presentación de la física. Por esto y por el énfasis en formalidad matemática, es que este libro
no resultará de lo más didáctico y recomendable para alguien que aborda muchos de los conceptos por primera vez. No obstante,
espero que sea de ayuda a alguien, y que alguien le pueda sacar tanto jugo como yo le saqué al escribirlo.

Si yo tuviera que enseñarle estos temas a alguien por primera vez, definitivamente no lo haría como lo hace este libro, que
apunta más a ir derecho al grano y a los bifes desde las definiciones lógicamente simples, pero invocadas como galerazos desconectados; sino
que propondría un programa basado fuertemente en experimentos, en la construcción gradual de un entendimiento en bases a leyes
parciales tangibles en casos particulares, para solo después, luego de haber trabajado mucho con los conceptos, llegar a las
definiciones y conceptos generales abstractos. Hacerlo de otra manera (por ejemplo como este libro lo hace) sería enseñar un modelo
matemático preciso de manera formal, pero no sería enseñar física, es decir, no sería una manera efectiva de ayudar a alguien a entender
el universo.

Esto último creo que es un problema que se encuentra bastante en la enseñanza de la matemática, donde muchas veces no es percibido
como un problema, porque total la matemática en sí es abstracta y no necesita motivaciones e intuiciones previas posiblemente erradas
o inferiores. Si bien eso es cierto de la matemática pura en sí misma, no es cierto que los humanos seamos máquinas abstractas desprovistas
de motivaciones, y por lo tanto un enfoque formalista radical en esas líneas es perjudicial para la educación matemática de los alumnos
\footnote{Lo cual no implica de ninguna manera que deba enseñarse la matemática de manera informal y falta de rigor, pues eso
ya dejaría de ser matemática para ser filosofía, divulgación de la ciencia, o alguna variante similar según el enfoque puntual.}.

Por último, luego de decir para qué no me parece útil este libro, me atrevo a decir para qué sí me parece de utilidad: Me parece de
utilidad como material de referencia, para consultar resultados y teoremas. También resulta de utilidad para alguien que quiera profundizar
o repasar los conceptos abordados, que no tenga miedo a la formalidad matemática, pero que ya tenga un cierto manejo de tales conceptos.
Finalmente, también puede servir para que alguien se haga una idea de cómo podría formalizarse la física, al menos en principio. Aunque
los temas de este libro son un resultado parcial, y están lejísimo de la imposible tarea (por la cantidad de trabajo necesario,
sin un beneficio claro e inmediato) de formalizar toda la física. Para los interesados en este último tema, recomiendo leer las opiniones de
Hilbert al respecto, y particularmente sobre su famoso sexto problema.

Otra advertencia que vale la pena hacer de antemano consiste en la terminología: Mi formación física es fuertemente autodidacta,
recopilando información de las más variadas fuentes principalmente en inglés. El resultado puede ser terminología o traducciones inusuales,
e incluso quizá ``incorrecta'' respecto de las convenciones aceptadas en algún caso (que esperemos sean los menos, ya
que no estoy enterado de ninguno).

En cuanto a su contenido, la presentación de los temas se hace en un orden inusual, basado en la formalidad matemática en
un sentido lógico. Se comienza con cantidades físicas, unidades y magnitudes, conceptos fundamentales para construir sobre ellos el análisis físico cuantitativo.
Luego se definen las nociones fundamentales del espaciotiempo, que nos proveerán un marco para las ideas que seguirán. Aquí se
comentan sutilezas y cuestiones varias sobre relatividad, para finalmente quedarnos con un espaciotiempo estilo física clásica,
deduciendo las transformaciones entre observadores inerciales a partir de las nociones básicas de simultaneidad y longitud absolutas.

Una vez que tenemos establecido sólidamente el espaciotiempo clásico, procedemos a construir la mecánica clásica, empezando
con cuestiones fundamentales sobre partículas, para luego estudiar el caso clásico de cuerpo rígido, fluidos, etc.

\subsection{Visión de la física}

¿Qué es la física? En realidad no es muy fácil responder satisfactoriamente a esta pregunta.

Hay varias maneras de concebir la física. Según wikipedia, la física consiste en el análisis general de la naturaleza,
llevado a cabo con el propósito de entender el comportamiento del universo. Esta es quizás la visión más acertada, porque
se centra en el propósito de la física. Y por lo tanto, explica por qué un físico puede hacer caso omiso a críticas de un matemático
cuando se dan diálogos como
``Esa demostración no está completa'', ``Es verdad... Sería mucho mejor si la completáramos, ojalá podamos algún día, pero por
ahora no se contradice con los experimentos, el argumento parece razonablemente convincente, y nuestra prioridad es entender
el universo, y este teorema sí que nos ayuda a entenderlo''.

Dicho esto, la matemática es una herramienta fundamental para entender el universo. Esto es, si se quiere, un accidente
\footnote{Aunque en realidad no lo es. Al estar nosotros, los seres humanos mismos que hacen física y matemática, inmersos en el universo,
la matemática que construímos y que nos interesa resulta naturalmente ser una herramienta importante para el estudio del universo.}, que algunos filósofos
llaman ``La irrazonable eficacia de las matemáticas''. Si hacemos abstracción, y suponemos que en efecto la física va a
proponer modelos matemáticos del universo con el afán de entenderlo entendiendo en su lugar esos modelos, podemos tomar
una visión extremadamente simplista de la física.

En efecto, podemos pensar que la física consiste de dos partes: En primer lugar, la física se encarga de proponer un modelo matemático
del universo, que capture su comportamiento. Para esto debe nutrirse de la observación y la experimentación, jueces últimos de
cualquier modelo, ya que si hay dos teorías en pugna y una se ajusta a los experimentos y la otra no, debemos quedarnos en física
con la que se ajusta a las observaciones y experimentos, aunque la otra sea mucho más linda, o esté mejor formulada matemáticamente,
o tenga cualquier otra ventaja aparente \footnote{Esto no quita que la teoría bonita pero errada no pueda servir como una excelente y muy útil
aproximación a la otra teoría mejor, de manera que siga valiendo la pena estudiarla. Pero esto debe estar justificado por la experiencia y
por las características matemáticas de las teorías. Por ejemplo, la teoría de Newton es matemáticamente una excelente aproximación
a la teoría de la relatividad especial, mientras no se manejen grandes velocidades. Como la teoría de la relatividad especial se
considera mejor por ajustarse bien a los experimentos, y la de Newton es una excelente aproximación a la misma cuando no hay
velocidades enormes, se considera útil y adecuada la teoría de Newton bajo estas hipótesis, y su estudio sigue ayudando muchísimo
a la compresión del universo, que es el objetivo final de la física.}.

En segundo lugar, la física consiste en el estudio matemático de estos modelos. Esta parte de la física no tiene ninguna diferencia
esencial con la matemática, y se puede decir directamente que es matemática. La única diferencia entre la matemática y la física
(o entre un físico y un matemático) en esta parte, consiste en el propósito con el cual lleva a cabo su actividad, y en
el tipo de objetos matemáticos que le interesarán estudiar (los cuales están fuertemente ligados al propósito por el cual llevan
adelante el mencionado estudio matemático).

Esta visión simplista de la física es la que tomaremos nosotros. En este sentido, construiremos matemáticamente esos modelos
de los que hablamos. Notar que al entender cuál es el objetivo de la física, queda claro por qué no le resulta importante a un físico
formalizar completamente y ser extremadamente precisos en la definición de estos modelos: Lo importante es entender el universo,
y cualquier esfuerzo de formalización adicional que no aporte a la comprensión del universo es desde un punto de vista físico tiempo
perdido, o al menos, es tiempo no utilizado en hacer física, y por lo tanto es una tarea que es mejor dejarle a los matemáticos.
A esto se suma la dificultad enorme de formalizar algo como el universo: Muchos modelos son solo descripciones parciales, a veces
distintos modelos son incompatibles entre sí, y por lo tanto es imposible obtener una formalización consistente que incorpore todos
esos modelos. Pero se puede pensar no obstante que este sería el fin último, el santo grial de la física: Obtener un modelo matemático
preciso que describa perfectamente el comportamiento del universo.

\pagebreak

\section{Magnitudes, Dimensiones, Cantidades y Unidades}

\subsection{Intuición inicial}

Daremos por entendido el concepto de magnitud, ya que no necesitaremos formalizarlo ni nos importará mucho, y será principalmente de uso
conceptual. Basta saber
que una magnitud es algo que puede ser medido cuantitativamente (tiempo, distancia, posición, presión, velocidad, humedad relativa, etc).
Y justamente una medición concreta de un valor para una magnitud se corresponde con una cantidad física específica, que
normalmente expresamos como un número y una unidad (al menos para el caso simple de magnitudes escalares), como por ejemplo
cosas de la pinta $\SI{3}{\meter}$ o $\SI{4,5}{\meter \per \second}$. Es decir, una magnitud será un aspecto de la realidad que se
puede medir cuantitativamente, y esas mediciones se corresponderán con objetos matemáticos precisos (escalares, vectores, matrices,
tensores, etc).

Las cantidades físicas son fundamentales en la física clásica. Básicamente constituyen la expresión de una magnitud escalar,
algo que se puede medir y caracterizar adecuadamente con un número real (teniendo fijado un sistema de unidades adecuado).
A continuación nos propondremos caracterizar este comportamiento de cantidades físicas, pero antes de eso empezaremos mostrando
un ejemplo concreto para entender intuitivamente a qué nos referimos luego. Considere el lector entonces lo siguiente como
una introducción informal, importante para interpretar más fácilmente las definiciones posteriores, pero lógicamente prescindible.

Consideremos por ejemplo tres \textit{dimensiones} básicas: \textit{masa}, \textit{longitud} y \textit{tiempo}. Tomaremos como unidades
las unidades del \textbf{SI} (\textbf{S}istema \textbf{I}nternacional de unidades) utilizadas para medirlas: El \textit{kilogramo} (``kilo'' para los amigos), notado como $\si{\kilogram}$,
para medir masa; el \textit{metro}, notado $\si{\meter}$, para medir longitud; y finalmente el \textit{segundo}, notado $\si{\second}$, para medir tiempo. Notar
que es convención estilística ampliamente aceptada escribir las unidades siempre en letra imprenta, al estilo mostrado, para no confundirlas con variables en ecuaciones
(por ejemplo, $3m$ puede referirse a algo como ``el triple de la masa $m$'', pero no puede ser ``tres metros'', que sería $\SI{3}{\meter}$).

El lector podrá ir a la verdulería y pedir $\SI{3}{\kilogram}$ de bananas, y le darán sus bananas, así que estará familiarizado con esto.
Lo mismo para longitudes y tiempos, uno sabe lo que significa que un corredor haga los $\SI{100}{\meter}$ en menos de $\SI{10}{\second}$.
Pero con solo medir masas, tiempos y longitudes no nos alcanza. Necesitaremos \textit{dimensiones derivadas}, que se obtengan a
partir de estas. Sin ir más lejos, el ejemplo del corredor invita claramente al concepto de \textit{velocidad}. El lector se dará cuenta de
que si el corredor corriese a velocidad constante, su velocidad sería de $\SI{10}{\meter\per\second}$ (que se lee ``diez metros por segundo'', dado que
\textit{por cada} segundo, el corredor \textit{recorre diez metros}). Y aquí nos encontramos
con una nueva dimensión, la velocidad, que resulta ser el cociente de la longitud y el tiempo, es decir es una dimensión derivada,
y su unidad, $\si{\meter \per \second}$, resulta obtenerse a partir de las unidades básicas $\si{\meter}$ y $\si{\second}$,
correspondientes a la longitud y al tiempo respectivamente.

Sin esforzarse mucho, el lector entenderá ahora que generalizando este ejemplo, podremos obtener cantidades para magnitudes derivadas de la pinta
$k \ \si{\meter}^{e_1}\si{\kilogram}^{e_2}\si{\second}^{e_3}$, con $k \in \mathbb{R}$ y $e_1,e_2,e_3 \in \mathbb{Q}$. El
ejemplo de $\SI{10}{\meter\per\second}$ corresponde a $k=10$, $e_1=1$, $e_2=0$ y $e_3=-1$.

El álgebra de unidades que pretendemos construir es muy intuitiva y sencilla. Solo podemos sumar o restar dos cantidades cuando sus
unidades resultan \textit{compatibles}, es decir, de la misma dimensión. De esta manera por ejemplo, sabiendo que
$\SI{100}{\centi \meter} = \SI{1}{\meter}$, tenemos que $\SI{2}{\meter}$ y $\SI{25}{\centi \meter}$ son cantidades compatibles, y su suma
será $\SI{2,25}{\meter}$ o equivalentemente, $\SI{225}{\centi \meter}$. En cambio no podemos sumar $\SI{3}{\meter}$ y $\SI{2}{\second}$,
ya que así como no podemos comparar (ni sumar) peras con roqueforts, no podemos mezclar un tiempo y una longitud de esta manera, ya
que son dimensiones diferentes y no hay una equivalencia entre ellas\footnote{En realidad se puede establecer una relación entre ellas,
pero esto no resta importancia al estudio de las unidades. Ver la sección que habla sobre sistemas de unidades naturales.}.

En cambio, es notable que siempre podemos multiplicar o dividir\footnote{Salvo el caso de la división por cero.} dos cantidades.
Simplemente multiplicamos los números y las unidades por separado de la manera obvia: 
$\SI{2}{\second} \cdot \SI{10}{\meter\per\second} = \SI{20}{\meter}$. Este ejemplo sencillo corresponde a una cuenta muy simple
que el lector debe ser capaz de hacer solito sin necesidad de saber un pomo de física: Una persona que cada segundo que pasa
recorre diez metros, si corre durante dos segundos recorre un total de veinte metros.

El ejemplo anterior mostraba una división: $\frac{\SI{100}{\meter}}{\SI{10}{\second}} = \SI{10}{\meter\per\second}$, con la
interpretación intuitiva antes mencionada. Como vemos, multiplicar y dividir cantidades no trae problemas, sean cuales sean
las unidades. El caso general se comportaría según la regla totalmente esperada:

$$k_1 \ \si{\meter}^{a_1}\si{\kilogram}^{a_2}\si{\second}^{a_3} \cdot k_2 \ \si{\meter}^{b_1}\si{\kilogram}^{b_2}\si{\second}^{b_3} =
 (k_1 \cdot k_2) \ \si{\meter}^{a_1 + b_1}\si{\kilogram}^{a_2+b_2}\si{\second}^{a_3+b_3}$$

Además, observe el lector que como dijimos antes, está bien definido cuándo una cantidad es mayor que otra siempre y cuando tengan
la misma dimensión, y en particular cuando una es positiva (mayor que cero). Entendido perfectamente todo esto, el lector ya entiende en un
sentido práctico lo que son las unidades y qué pretendemos hacer con ellas. Es esta estructura intuitiva la que vamos a formalizar
en la sección que sigue.

\subsection{Descripción de las dimensiones físicas}

\begin{definicion}
    Un \textbf{espacio de dimensiones} es una tupla $(\Dimensions,\cdot_d, \circ, \cdot_q)$ que cumple todo lo siguiente:
    
    1) Cada elemento de $\Dimensions$ es un \rev de dimensión $1$ (con una operación suma que notamos con $+$) que llamaremos una dimensión, y son todos disjuntos. Si $D \in \Dimensions$, $v \in D$, diremos que $D$ es la dimensión de $v$,
    y lo notaremos $d(v) = D$
    
    2) $(\Dimensions, \cdot_d, \circ)$ es un \qev, con suma de vectores $\cdot_d$ y acción de $\mathbb{Q}$ en $\Dimensions$ dada por $\circ$.
    Al vector cero de $\Dimensions$ lo llamamos la dimensión adimensional\footnote{Es un nombre un tanto contradictorio, pero como veremos esta dimensión representa simples reales sin una unidad, un número y ya, o sea, una cantidad adimensional.}.
     
    3) Notamos $\Quantities$ al conjunto unión de todos los elementos de $\Dimensions$. A los elementos de $\Quantities$ los llamaremos cantidades físicas. $(\Quantities, \cdot_q)$ es un monoide conmutativo
    sin divisores de $0$, y compatible con la estructura de de $\Dimensions$. Más precisamente, esto último significa:
    
    $$d(a \cdot_q b) = d(a) \cdot_d d(b) \ \  \forall a,b \in \Quantities$$
    
    $$(a+b) \cdot_q c = a \cdot_q c + b \cdot_q c \ \  \forall a,b \in D \in \Dimensions, \ c \in D' \in \Dimensions$$
    
    $$(\lambda a) \cdot_q b = \lambda (a \cdot_q b) \ \ \forall \lambda \in \mathbb{R}, \ a,b \in \Quantities$$
    
    $$a \cdot_q b = 0 \Rightarrow a = 0 \lor b = 0 \ \ \forall a,b \in \Quantities$$
    
\end{definicion}

Notemos que a la acción $\circ$ la notamos simplemente con la yuxtaposición, como es usual para la acción de un escalar sobre un vector.
Interpretemos rápidamente esta definición en términos de la intuición dada previamente. 

Usamos notación multiplicativa para la suma de vectores en $\Dimensions$, ya que esta suma de vectores corresponde a
multiplicar dimensiones, y la acción es elevar una dimensión a un racional. Por otro lado los elementos de $\Quantities$ son
justamente todas las \textit{cantidades físicas} posibles, y para cada una de ellas $q \in \Quantities$, $d(q) \in \Dimensions$
es su dimensión correspondiente. 

La suma de vectores en $D \in \Dimensions$ es simplemente sumar cantidades físicas, lo cual
solo tiene sentido como vimos antes, si estamos sumando cantidades de una misma dimensión. El producto $\cdot_q$ corresponde
al producto de cantidades físicas (esas que damos con número y unidad), y el $\cdot_d$ corresponde al producto de dimensiones,
o sea, eso que hacemos cuando decimos que multiplicar una velocidad por un tiempo da por resultado una longitud.

De ahí resulta que la primer condición que pedimos al monoide $(\Quantities, \cdot_q)$ expresa que la dimensión del producto
de cantidades es el producto de las dimensiones correspondientes, o sea que las dimensiones se pueden manejar por un lado
independientemente del valor exacto de la cantidad, cosa que es fácil observar en los ejemplos anteriores intuitivos. Notar
que gracias a esta condición, en la segunda expresión resulta que el lado derecho está bien definido, pues la suma es
entre cantidades de la misma dimensión ($a \cdot_q c$ y $b \cdot_q c$ tienen la misma dimensión, pues ambas tienen dimensión
$D \cdot_d D'$ en virtud de la primer propiedad).

Por otra parte, las segundas dos condiciones expresan que $f_{D,a} : D \rightarrow d(a) \cdot_d D$, dada por 
$f(b) = a \cdot_q b$ es una transformación lineal (homomorfismo)
de $D$ en $d(a) \cdot_d D$, para cada $D \in \Dimensions$ y $a \in \Quantities$. En este sentido el producto preserva la estructura de espacio
vectorial de las cantidades, y al ser espacios de dimensión $1$, multiplicar por una cantidad fija termina siendo esencialmente
como multiplicar por una constante (la matriz en cualesquiera bases de $f_{D,a}$ es una matriz de $1 \times 1$).

Todas estas observaciones deberían ayudar a ir entendiendo como esta definición formal captura la intuición del ejemplo de la sección anterior.
Estudiaremos ahora algunas propiedades de esta estructura, y finalmente veremos que en efecto se corresponde perfectamente con la intuición anterior.
En lo sucesivo notaremos a $\cdot_d$ y a $\cdot_q$ simplemente con $\cdot$. Esto no generará confusión ya que se puede deducir cuál
de ellos aplicamos de acuerdo a los elementos a los que lo aplicamos (uno opera con cantidades físicas y el otro con dimensiones), y hemos venido distinguiéndolos solo para mantener una
prolijidad inicial hasta entrar en confianza. Eventualmente, directamente dejaremos de utilizar el $\cdot$ para reemplazarlo
por la simple yuxtaposición de los factores involucrados (aunque por unos párrafos más, usaremos yuxtaposición para la acción de $\mathbb{R}$ en una dimensión y
el $\cdot$ para los productos en $\Dimensions$ y $\Quantities$).

Notemos en primer lugar que el elemento neutro del monoide $1 \in \Quantities$, es una cantidad adimensional, es decir, $d(1) = 1$
(entendiendo el segundo $1$ como el vector nulo en $\Dimensions$).
En efecto, $d(1) = d(1 \cdot 1) = d(1) \cdot d(1)$, y recordando que $(\Dimensions, \cdot)$ es un grupo (por ser los vectores de un \qev),
podemos cancelar y resulta $d(1) = 1$. Además no puede ser $1 = 0$ (siendo el segundo el $0$ de la dimensión adimensional), pues
sería $v = 1 \cdot v = 0 \cdot v = (0 0) \cdot v = 0 (0 \cdot v) = 0$ para cualquier $v \in \Quantities$, un claro absurdo.

Notemos ahora que $(\lambda 1) \cdot v = \lambda (1 \cdot v) = \lambda v$, de manera que si identificamos a $\lambda 1$ con $\lambda \in \mathbb{R}$,
la acción $\lambda v$ resulta equivalente al producto del asociado $(\lambda 1) \cdot v$. Luego es natural identificar estos elementos
entre sí, de manera que consideramos a $\mathbb{R}$ como la dimensión adimensional, y entonces las cantidades físicas resultan
extender y generalizar el cuerpo de los reales. Vemos que como habíamos dicho, las cantidades adimensionales resultan ser números
reales comunes, sin ninguna unidad asociada. Es claro además que con esta identificación, el producto usual en $\mathbb{R}$ coincide
con la restricción de nuestro $\cdot$ al caso adimensional, pues $(a 1) \cdot (b 1) =  (ab)1$, así que tanto la suma como el producto
en $\Quantities$ extienden a los respectivos de $\mathbb{R}$, y la acción de $\mathbb{R}$ sobre elementos de $\Quantities$ coincide
con el producto extendido.

Miremos ahora los ceros por un momento. Notemos que no hay \textit{un} cero, sino muchos ceros, uno por cada dimensión, y son
elementos diferentes de $\Quantities$. Venimos usando el mismo símbolo $0$ para referirnos a todos ellos, pero esto no debería
generar confusión, ya que allí donde hay un $0$ se interpreta como un $0$ de la dimensión adecuada al contexto.

Otra propiedad sencilla pero importante es que $a \cdot b = 0 \Leftrightarrow a = 0 \lor b = 0$. Esto equivale a que
$f_{D,a}$ es la transformación lineal nula exactamente cuando $a = 0$ (recordar que como la dimensión de los espacios es $1$,
si no es nula es inversible). Ya hemos mostrado que $a = 0 \Rightarrow f_{D,a} = 0$. La vuelta es exactamente nuestra
hipótesis de que el monoide $(\Quantities, \cdot_q)$ no tiene divisores de $0$. En particular el producto de elementos no nulos
resulta en elementos no nulos.

Gracias a esta propiedad podemos demostrar que los elementos no nulos de nuestro monoide forman en realidad un grupo con el producto, es decir
que habrá inversos y por lo tanto podemos dividir además de multiplicar, siempre que no dividamos por cero. Para esto basta notar
que si tenemos una cantidad no nula $q \in \Quantities$ con $d(q) = D \in \Dimensions$, aprovechando la estructura de \qev de $\Dimensions$
existe una $D^{-1} \in \Dimensions$, y si tomamos un $q' \in D^{-1}$ no nulo se tiene que $d(q q') = 1$, es decir
$q q' = \lambda \in \mathbb{R}$ con $\lambda \neq 0$ (pues $q$ y $q'$ son no nulos). De ahí que $\frac{1}{\lambda} q'$ es un
inverso para $q$.

Recapitulando, tenemos entonces un conjunto de cantidades físicas $\Quantities$, con suma y producto bien definidos y particionado
según $\Dimensions$, de forma que solo podemos sumar y restar cantidades de una misma dimensión, pero siempre podemos multiplicar
dos cantidades o dividir por una cantidad no nula, y además $\mathbb{R} \subseteq \Quantities$ y las operaciones en $\Quantities$
extienden naturalmente las operaciones respectivas en $\mathbb{R}$. 

Nos falta definir ahora una noción de orden, es decir,
queremos comparar cantidades de la misma magnitud. Como cada espacio vectorial de dimensión $1$ tiene dos ordenaciones posibles
compatibles con la estructura de \rev, bastará fijar los elementos positivos. Para ello definimos que un elemento
$q \in \Quantities$ es no negativo, y lo notamos $q \geq 0$, si es un cuadrado, o sea existe $q' \in \Quantities$ tal que
$q = q'^2$. Decimos que $q$ es positivo, $q > 0$, si es no negativo y no nulo.

Notemos en primer lugar que la positividad induce un orden en cada dimensión. En efecto, dada una dimensión $D \in \Dimensions$,
aprovechando la estructura de \qev de $\Dimensions$ tenemos que para que $u^2 \in D$, debe ser necesariamente $u \in D^{\frac{1}{2}}$,
y tomando un $u \in D^{\frac{1}{2}}$ no nulo, resulta que los cuadrados de $D$ son exactamente los
$(\lambda u)^2 = \lambda^2 u^2$. Como $u^2 \in D$ no es nulo, estos elementos corresponden a una semirrecta en $D$.
Es claro que la relación en $D$ dada por $v \geq w \Leftrightarrow  v - w \geq 0$ es una relación de orden, y es la única
posible si pretendemos que los cuadrados sean positivos. Luego ya tenemos definida una relación de orden con la que podemos
comparar elementos de una misma dimensión.

Notemos que el producto satisface la regla de los signos: Si multiplicamos positivos obtenemos positivos, pues es claro que
al multiplicar cuadrados no nulos obtenemos cuadrados no nulos. El resto de las reglas salen de esta, por ejemplo,
un negativo por un positivo da un negativo, ya que un negativo es, por definición, el inverso aditivo de un positivo,
luego tal producto resulta ser el inverso aditivo de un producto de positivos, y por lo tanto negativo. Similarmente
con dos negativos tendremos el inverso del inverso, que será el valor positivo inicial.

De estas propiedades también surgen inmediatamente las usuales para desigualdades respecto del producto, es decir,
multiplicar una desigualdad por un positivo a ambos lados preserva la desigualdad, y multiplicarla por un negativo
la invierte. Las propiedades de las desigualdades respecto de la suma son las usuales inmediatamente, pues cada
\rev con el orden resulta isomorfo a $\mathbb{R}$ (con la salvedad de que no hay un elemento ``$1$'' destacado, pero
esto no es importante para el orden).

Para finalizar, nos queda por definir la noción de elevar una cantidad positiva a un número racional. Usando un argumento
análogo al que utilizamos para definir las cantidades positivas como los cuadrados, y sabiendo que en $\mathbb{R^+}$ podemos tomar
raíces enésimas, es fácil ver que para cada $q \in \Quantities$
positivo, existe un único $q' \in \Quantities$ positivo y tal que $q'^n = q$, para cada natural $n$, es decir, podemos tomar
raíces enésimas a cantidades positivas. Podemos definir así $q^{\frac{p}{n}} = q'^p$ y $q^{-\frac{p}{n}} = \left (\frac{1}{q'} \right)^p$. Es fácil ver que esta potenciación
está bien definida (representaciones distintas del exponente $r = \frac{p}{n}$ dan lugar al mismo valor $q^r$),
distribuye con el producto, que encadenar potenciaciones equivale a multiplicar los exponentes, que
$d(q^r) = d(q)^r$, etc, ya que todas estas demostraciones son análogas a lo que se hace cuando se define la exponenciación
de un real positivo a un valor racional.

Con esto ya tenemos definidas las nociones fundamentales para trabajar con dimensiones, que, como vemos, no es más que
capturar la noción intuitiva de hacer cuentitas con unidades en una estructura matemática que lo refleje.

\subsection{Descripción constructiva de las dimensiones: Unidades}

La sección anterior no es constructiva para nada. Definimos una noción de espacio de dimensiones a partir de unas propiedades
básicas, y luego vemos que tiene
una estructura muy prolija que nos agrada, pero ni siquiera hemos mostrado un modelo de tal espacio de dimensiones, con
lo cual a priori ni siquiera tenemos probada la consistencia de tal noción. Además, no tenemos una idea concreta de como
operar con cantidades a partir de su definición. Eso pretendemos hacer en esta sección: Unificar la definición teórica,
abstracta, general, de la sección anterior, con la noción concreta de ``numerito $+$ unidad'' que venimos manejando
intuitivamente.

Para ello vamos a definir el siguiente concepto fundamental:

\begin{definicion}
Un \textbf{conjunto de dimensiones fundamentales} para un espacio de dimensiones es una base del \qev $(D, \cdot)$.
\end{definicion}

Esta definición es sencilla pero encierra un concepto sutil...

Volvamos al ejemplo inicial intuitivo, en el cual considerábamos las cantidades de la pinta
$k \ \si{\meter}^{e_1}\si{\kilogram}^{e_2}\si{\second}^{e_3}$, con $k \in \mathbb{R}$ y $e_1,e_2,e_3 \in \mathbb{Q}$.
Tenemos que una dimensión es para nosotros una combinación de tiempo, longitud, y masa. Claramente estas tres forman una base
del espacio de dimensiones que manejamos en este ejemplo, y resultan ser lo que acabamos de llamar un conjunto de dimensiones
fundamentales. Ahora bien, la noción de base en un espacio viene de la mano de la noción de \textit{cambio de base}: ¿Se
podrán tomar otras dimensiones fundamentales en el ejemplo?

La respuesta resulta ser que sí. Por ejemplo podemos tomar como base, longitud, velocidad y masa. Así, tiempo ahora no es
una dimensión fundamental sino que se obtiene como longitud sobre velocidad. Sin ir más lejos podemos imaginarnos que
tomamos como unidades al $\si{\meter}$, al $\si{\meter\per\second}$ y al $\si{\kilogram}$ y tendremos cantidades de la
pinta
$k \ \si{\meter}^{e_1}\si{\kilogram}^{e_2}\si{\meter\per\second}^{e_3}$, con $k \in \mathbb{R}$ y $e_1,e_2,e_3 \in \mathbb{Q}$.
El isomorfismo (cambio de base) entre esta representación y la anterior es claro y resulta de hacer una simple cuentita
con los $e_1, e_2, e_3$.

Este ejemplo muy sencillo da cuenta de un concepto muy importante, que es que la elección de dimensiones fundamentales
es completamente arbitraria. El SI determina usar masa, tiempo y longitud como unidades fundamentales, pero por ejemplo
el sistema técnico es un sistema alternativo que toma como dimensiones fundamentales del mismo espacio al tiempo,
la longitud y la fuerza, es decir, reemplaza masa por fuerza. Ambos son completamente equivalentes y se corresponden
por medio de un sencillo cambio de base.

Recapitulando, un conjunto de dimensiones fundamentales nos permitirá expresar cada dimensión como un producto de las dimensiones
fundamentales elevadas a exponentes racionales adecuados, de forma única. ¿Podremos obtener una expresión similar para cantidades
físicas?

\begin{definicion}
Un \textbf{sistema de unidades fundamentales} para un espacio de dimensiones es un conjunto $U \subseteq \Quantities$ de cantidades positivas, de manera que
no haya dos de la misma dimensión y el conjunto $\{d(u) | u \in U \}$ sea un conjunto de dimensiones fundamentales para el mismo espacio
de dimensiones.
\end{definicion}

En otras palabras, un sistema de unidades fundamentales se obtiene tomando una cantidad positiva de cada una de de las dimensiones
de un conjunto de dimensiones fundamentales. Así como vimos la total arbitrariedad del conjunto de dimensiones fundamentales,
queda evidentemente claro que el sistema de unidades fundamentales es igualmente arbitrario. 

La importancia de los sistemas de unidades radica en el siguiente

\begin{teorema}
Dado un sistema de unidades fundamentales $U$, para cada elemento $q \in \Quantities$ existen un único $U' \subseteq U$ finito,
una única función $e : U' \rightarrow \mathbb{Q} \setminus \{0\}$ y $k \in \mathbb{R}$ de manera que
$$q = k \prod_{u \in U'}{u^{e(u)}}$$
\end{teorema}

\begin{proof}
Tomando dimensiones queda claro que debe ser $d(q) = \prod_{u \in U'}{d(u)^{e(u)}}$. Como los $d(u)$ forman un conjunto de 
dimensiones fundamentales, existen únicos $U'$ finito y $e$ que cumplen esto (recordar que formar un conjunto de dimensiones
fundamentales es simplemente ser una base del espacio de las dimensiones). Como los $u \in U$ son positivos, esto determina
que $\prod_{u \in U'}{u^{e(u)}}$ está bien definido, es positivo y tiene la dimensión correcta, luego existe un único
$\lambda$ que cumple lo pedido (ya que al ser no nulo, tal valor resulta ser una base de $d(q)$).
\end{proof}

La importancia de este teorema radica en que nos caracteriza completamente la forma de los elementos de $\Quantities$, y
nos asegura que al fin y al cabo, cualquier espacio de dimensiones resulta isomorfo al ejemplo intuitivo que dimos antes.

Dicho sea de paso, es fácil verificar que tales expresiones constituyen un ejemplo de espacio de dimensiones ya que cumplen
todos los axiomas de la definición, luego son los únicos. Particularmente se desprende que dos espacios de dimensiones cualesquiera
tienen esta pinta, y por lo tanto son isomorfos siempre y cuando los $\Dimensions$ correspondientes tengan la misma dimensión
como \qev.

Otra consecuencia importante del teorema es que nos provee una representación práctica, en el sentido de que es muy fácil operar
con expresiones representadas en un sistema de unidades prefijado. Concretamente, es inmediato verificar que la expresión de
un producto resulta de sumar exponentes y multiplicar los escalares: Y la expresión de la suma resulta simplemente de sumar
los escalares, y la suma está definida exactamente cuando los exponentes coinciden, ya que los exponentes determinan completamente
la dimensión de la cantidad en cuestión.

La elegancia de construir de esta manera axiomática los espacios de dimensiones en lugar de partir de un conjunto de unidades
fundamentales prefijado radica justamente, en que tales unidades son arbitrarias, y tal arbitrariedad queda clarísima con
nuestra definición, mientras que si hubiéramos partido de un conjunto de unidades prefijado y construído directamente
el conjunto de las cantidades, no sería tan claro que esas unidades particulares elegidas no tienen un rol privilegiado.

Si bien hemos hablado de ``un'' espacio de dimensiones, en el sentido de definición matemática general y suponiendo que puede haber
muchos (y los hay, ya que hemos explicado como construírlos explícitamente), lo cierto es que para nuestro estudio físico nunca
trabajaremos con varios a la vez. En particular, nos conformaremos con pensar que ya existe allí un (``el'') espacio de dimensiones, con
tantas dimensiones como necesitemos, y en todo el libro trabajaremos con él. Cuando necesitemos una dimensión nueva la introduciremos,
y esto no trae problemas pues podemos pensar que siempre estuvo allí en nuestro espacio de dimensiones, solo que nunca la habíamos
mencionado antes.

De esta manera, cuando hablemos de una dimensión, o una cantidad física, se entenderá que nos referimos a los elementos de
$\Dimensions$ o de $\Quantities$, conjuntos que están fijos a lo largo de todo el libro y corresponden a un espacio de dimensiones
existente y adecuado para la descripción de la física que haremos (o sea, que tenga todas las dimensiones que vayamos a usar).

\subsection{Espacio vectorial dimensional}

Con esto que hemos construído tenemos las cantidades físicas escalares bien representadas,
pero ¿Qué hay de los vectores? Concretamente, nos gustaría
poder manejar expresiones como

$$\SI{5}{\kilogram \per \meter} (\SI{3}{\meter}, \SI{2}{\meter}, \SI{-4}{\meter} ) = (\SI{15}{\kilogram}, \SI{10}{\kilogram}, \SI{-20}{\kilogram} )$$

Además, no queremos tener que hacer una construcción particular fijando bases con cada espacio vectorial con el que queramos trabajar, así que para
ello definiremos las siguientes nociones destinadas a capturar el comportamiento de las dimensiones en lo que a vectores respecta.

\begin{definicion}
Un \textbf{$\mathbb{R}$-espacio vectorial dimensional} es una tupla $(\mathcal{V_D}, \circ)$ que cumple todo lo siguiente:

1) $\mathcal{V_D}$ es un conjunto de $\mathbb{R}$-EV disjuntos, de manera que $\mathcal{V_D}$ sea un $\mathbb{Q}$-espacio afín cuyo espacio vectorial asociado sea $\Dimensions$.
     Dado $V \in \mathcal{V_D}$ y $D \in \Dimensions$, notamos $D \cdot V$ a la traslación correspondiente en $\mathcal{V_D}$, que traslada al punto $V$ según el vector $D$.

2) Llamamos $\mathcal{V}$ a la unión de todos los elementos de $\mathcal{V_D}$. Dado un $v \in V \in \mathcal{V_D}$, notaremos como
$e(v) = V$.

3) $\circ : \Quantities \times \mathcal{V} \rightarrow \mathcal{V}$ es una acción compatible con la estructura de $\mathcal{V}$, o sea:

   $$ e(qv) = d(q) \cdot e(v) \ \ \forall q \in \Quantities, v \in \mathcal{V} $$

   $$ q(v+w) = qv + qw  \ \ \forall q \in \Quantities, v,w \in V \in \mathcal{V_D}$$
   
   $$ (q+q')v = qv + q'v  \ \ \forall q,q' \in D \in \Dimensions, v \in \mathcal{V}$$
   
   $$ (qq')v = q'(qv)  \ \ \forall q,q' \in \Quantities, v \in \mathcal{V}$$

   $$ \circ \mbox { coincide con la acción de } \mathbb{R} \mbox{ en } V \ \ \forall V \in \mathcal{V_D}$$

\end{definicion}

La intuición es cada cada elemento $V \in \mathcal{V_D}$ es un espacio vectorial de vectores con una dimensión diferente,
de manera que al multiplicar una cantidad por un vector, obtenemos un nuevo vector pero que puede caer en otro espacio
vectorial, si cambia la dimensión. Si la cantidad física era adimensional no cambiará la dimensión, porque la última
condición nos garantiza que en ese caso el efecto es el mismo que la acción normal cuando la cantidad se considera como
un número real.

Caractericemos rápidamente la estructura de esta extensión que hemos hecho para ver que la hemos hecho bien.
Notemos que lo que hemos pedido es muy razonable: que $\circ$ extienda la acción de $\mathbb{R}$ en cualquiera de los $V$, a una
acción general de $\Quantities$ en $\mathcal{V}$ que cumpla con las propiedades usuales de la acción de un escalar sobre un espacio
vectorial y respete las dimensiones.

Observemos que la primer condición, que se respeten las dimensiones, garantiza que las sumas en las otras expresiones están todas
bien definidas (son sumas entre entidades de la misma dimensión, que en el caso de los vectores se entiende como que son parte del mismo
espacio vectorial, y por lo tanto se pueden sumar).

Consideremos, similarmente a como hicimos antes, la $f_{V,q} : V \rightarrow d(q) V$, dada por 
$f(v) = qv$, donde $V \in \mathcal{V_D}$ y $q \in \Quantities$. De las últimas dos propiedades se
deduce que $f_{V,q}$ saca afuera escalares, y de la segunda que se distribuye sobre la suma. Por lo
tanto $f_{V,q}$ resulta ser siempre una transformación lineal. Además de la anteúltima propiedad se
deduce que si $q \neq 0$, $f_{V,q}$ y $f_{d(q) V, q^{-1}}$ son inversas, por lo tanto cuando elegimos
un $q$ no nulo la $f$ resulta isomorfismo, y cuando $q = 0$ resulta $f_{V,q} = 0$.

Notemos que hemos probado la existencia de un isomorfismo entre cualquier par de elementos $V_1, V_2 \in \mathcal{V_D}$.
(concretamente, dado cualquier $q \in \frac{V_2}{V_1}$ no nulo, $f_{V_1, q}$ resulta ser isomorfismo entre $V_1$ y $V_2$.
El cociente entre $V_2$ y $V_1$ se entiende según nuestra notación como la resta de puntos en el espacio afín, que da
un vector del espacio asociado, es decir, un elemento de $\Dimensions$).
En particular existe un cardinal $n$ tal que $dim(V) = n$ para todos los $V \in \mathcal{V_D}$.

En general esta estructura se porta como esperaríamos: Igual que un espacio vectorial común, pero con unidades. Una salvedad
es que, aunque pueda parecerlo, no tenemos que $\mathcal{V}$ sea un $\Quantities$-espacio vectorial, ya que eso no tiene sentido
al no tener $\Quantities$ una suma definida siempre. Ni siquiera se puede decir que $\mathcal{V}$ sea un $\mathbb{R}$ espacio vectorial,
ya que la suma en $\mathcal{V}$ tampoco está definida para cualquier par de vectores. Esto implica en particular que no tendremos
una base de $\mathcal{V}$ en el sentido usual... Pero casi.

Notemos para esto que en un espacio vectorial dimensional $\mathcal{V}$, no hay ningún elemento $V$ privilegiado,
y no podemos hablar de la dimensión exacta que tienen los elementos de un $V$. Solo podemos hablar de la dimensión del
``cociente'' entre dos elementos, gracias a la estructura afín. En otras palabras, si tenemos un cierto vector de un espacio,
y lo multiplicamos por una cantidada $q$ de dimensión $D$, sabemos que irá a parar a un elemento de $D \cdot V$, con lo cual
el cociente de dimensiones entre esos espacios es $D$, pero no hablamos a priori de la dimensión de un elemento $V$ en forma absoluta.
Esto nos permite inmediatamente sumergir a cualquier espacio vectorial que tengamos en un espacio vectorial dimensional,
pues todos los elementos del mismo son indistinguibles. En particular, si tenemos vectores, podemos multiplicarlos por
cantidades físicas sin problema, y sabemos de lo que estamos hablando, pues podemos pensar que hemos sumergido el
espacio vectorial original en esta estructura.

Un caso particular de interés resulta cuando sí podemos asignar dimensiones concretas a los elementos de $\mathcal{V_D}$. Esto
se corresponde a fijar un origen en la estructura afín de $\mathcal{V_D}$, que jugará el rol de los vectores adimensionales.
Cuando así sea, diremos que estamos ante un espacio vectorial dimensional \textit{con dimensiones}. De cualquier manera,
damos nombres a estas definiciones formales por completitud, pero no tendremos que mencionarlas en la práctica, ya que su
valor radica en garantizar la validez y el marco formal para expresiones como la que hemos propuesto antes, donde multiplicamos
escalares y vectores.

En el caso de un espacio vectorial dimensional con dimensiones, notamos $V_1$, al unico elemento adimensional de $\mathcal{V_D}$,
correspondiente a los ``vectores adimensionales'', es decir, al origen que hemos fijado en la estructura afín. Esto asigna una dimensión concreta
a cada elemento de $\mathcal{V_D}$, correspondiente al vector desde el origen a tal punto en la estructura afín. De esta manera, notamos
$V_D$ al punto con vector desde el origen $D$, o sea los $v$ de la forma $q v$, con $d(q) = Q$ y $v \in V_1$. Además
podemos hablar de la dimensión de un $v \in V_D \in \mathcal{V_D}$, siendo esta $d(v) = D$, la dimensión de su espacio contenedor.

La conveniencia del caso en que tenemos fijado el $V_1$ radica en el siguiente

\begin{teorema}
Si $B$ es una base de $V_1$, entonces para cada $v \in \mathcal{V}$, existen un único $S \subseteq B$ finito y una $f : S \rightarrow d(v) \setminus \{0\}$ tales que:

$$ v = \sum_{u \in S}{f(u) u} $$
\end{teorema}
\begin{proof}
Llamamos $D = d(v)$, y tomamos $q \in D$ no nulo. Basta probar que existe una única escritura de la forma
$$ v = \sum_{u \in S}{f'(u) (qu)} \ \ f' : S \rightarrow d(1) \setminus \{0\}$$

Ya que multiplicando o dividiendo por $q$ a traves de la transformación $f(u) = qf'(u)$,
podemos biyectar ambos tipos de expresiones de $v$.

Consideramos la $f_{V_1, q}$. Como es isomorfismo, envía la base $B$ de $V_1$ a una base $B'$ de $V_D$. Luego es inmediato que
existe una única forma de expresar $v$ como combinación lineal de los $qu$, como queríamos.
\end{proof}

Luego esto nos da una forma concreta de expresar y manipular los elementos de $\mathcal{V}$. Si por ejemplo tuviéramos una
base de $V_1$, $B = {v_1,v_2,v_3}$, entonces podríamos expresar de manera única cada elemento a través de sus coordenadas, es decir,
tenemos identificado $\mathcal{V}$ con los elementos de $\Quantities^3$ tales que todas sus componentes tienen la misma dimensión.
Así podríamos manipular vectores de la manera intuitiva con expresiones como
$\frac{2}{\SI{3}{\meter \second}}(\SI{3}{\meter},0,\SI{2}{\meter}) + 
  (\SI{2}{ \per \second },\SI{3}{ \per \second },\SI{4}{ \per \second }) = (\SI{4}{ \per \second },\SI{3}{ \per \second } ,\frac{16}{3}\si{ \per \second })$.
Notar que al igual que pasaba con los espacios de dimensiones, dos espacios vectoriales dimensionales son isomorfos, fijada la dimensión de sus espacios,
dado que todos ellos resultan isomorfo a una construcción como en el caso de $\Quantities^n$ que hemos mostrado, construcción que es
fácil ver que cumple todas las hipótesis para ser un espacio vectorial dimensional \footnote{Más precisamente un espacio vectorial dimensional con dimensiones, pero esta distinción entre el caso afín y el espacio vectorial no presenta dificultades, y es simplemente fijar dimensiones concretas para los vectores cuando así interesa hacerlo}.

En particular, como los espacios vectoriales dimensionales en los que puede estar embebido un espacio vectorial son únicos salvo isomorfismo, podemos considerar que extendemos
un espacio vectorial cuando queramos, así podemos trabajar multiplicando cantidades físicas por vectores libremente, tal y como habíamos prometido hace algunos párrafos.
La gracia es que siempre que tengamos un $\mathbb{R}$-espacio vectorial $V$ y queramos extender su acción a una acción de $\Quantities$ en
$V$, podemos hacerlo obteniendo un nuevo espacio más grande, de manera de poder interpretar adecuadamente el producto resultante.
Cuando además sea de utilidad fijar bases y computar con coordenadas, con solo fijar cuál es la dimensión que tienen los vectores de $V$,
quedará determinada una estructura de espacio vectorial dimensional con dimensiones, y podremos usar la construcción con coordenadas
fijando una base de $V_1$ como hemos mostrado.

\subsection{Espacio vectorial dimensional con producto interno}

Lo que hemos hecho hasta ahora es formalizar cómo trabajar con unidades en el caso de espacios vectoriales o afines. Ahora bien,
como en física es fundamental la noción de longitud, y la estructura geométrica del espacio (o más en general del espaciotiempo),
tiene sentido que queramos trabajar con espacios vectoriales con producto interno. Entonces simplemente definimos:

\begin{definicion}
Un \textbf{espacio vectorial dimensional con producto interno} es un espacio vectorial dimensional con dimensiones, en el cual
existe una operacion de producto interno $\cdot : \mathcal{V} \times \mathcal{V} \rightarrow \Quantities$, con las siguientes caracteristicas:

$$v \cdot w \in \mathbb{R} \ \ \forall v,w \in V_1$$
$$(kv) \cdot w = k (v \cdot w) \ \ \forall k \in \Quantities \ \forall v,w \in \mathcal{V}$$
$$v \cdot w = w \cdot v \ \ \forall v,w \in \mathcal{V}$$
$$(u+v) \cdot w = u \cdot w + v \cdot w  \ \ \forall u,v \in V_D \in \mathcal{V_D} \ \forall w \in \mathcal{V}$$
$$v \neq 0 \Rightarrow v \cdot v > 0 \ \ \forall v \in \mathcal{V}$$
\end{definicion}

Es decir, es una generalización natural de la noción de espacio vectorial con producto interno común, para que tenga dimensiones.
En efecto, la primer condición asegura que el producto interno de dos vectores adimensionales es un simple real (una cantidad
adimensional), con lo cual $V_1$ con $\cdot$ formará un espacio vectorial con producto interno en el sentido común.

A su vez, esta condición junto a la segunda, garantiza que $d(v \cdot w) = d(v) d(w)$ (recordar el teorema de escritura en una
base de la sección anterior). Esto garantiza la suma en el miembro derecho de la cuarta ecuación está bien definida, pues
coinciden las dimensiones de los sumandos.

La última ecuación permite definir la norma de un vector: $||v|| = \sqrt{v \cdot v}$, y notar que de lo ya mencionado
se deduce que $d(||v||) = d(v)$. Una \textbf{base ortonormal} del espacio vectorial dimensional con producto interno,
es una base ortonormal de $V_1$ en el sentido usual. Notar que de las propiedades dadas se puede deducir que dada la
escritura en una tal base de dos vectores, su producto interno (en el sentido extendido con dimensiones) se calcula
como se calcula el usual, simplemente sumando los productos de las componentes correspondientes.

Así el cálculo es como siempre, y por ejemplo $(\SI{3}{\meter}, \SI{2}{\meter}) \cdot (\SI{-1}{\per \second}, \SI{2}{\per \second}) = \SI{1}{\meter \per \second}$

Si nos interesa el estudio de alguna geometría más exótica que la euclídea (por ejemplo, para estudiar relatividada especial),
podemos hacer la misma definición pero obviando la exigencia de que la forma bilineal dada por $\cdot$ sea definida positiva (última ecuación).
Todo funciona de la misma manera, y obtenemos un espacio vectorial con dimensiones y una forma bilineal asociada. Nociones como
la signatura de la forma bilineal no se ven alteradas (en particular, restringido a $V_1$ la forma bilineal no tiene nada de nuevo,
y extenderla no cambia la signatura).

\subsection{Análisis dimensional}

El fin del análisis dimensional consiste simplemente en analizar las dimensiones involucradas en ecuaciones físicas, y asegurarse
que las operaciones con cantidades físicas respetan las
dimensiones. Así, el uso más básico de análisis dimensional consiste simplemente en calcular las dimensiones a ambos lados de
una ecuación y verificar que coincidan (ya que de no ser así, tendríamos una expresión sinsentido), y similarmente, verificar
que las operaciones tengan sentido (Por ejemplo tomar logaritmo a una longitud no tiene sentido al no tener una unidad fija,
solo podemos tomar logaritmo a números reales. Similarmente, si sumamos o restamos dos cantidades debemos asegurarnos que
sus dimensiones coincidan). Este uso del análisis dimensional funciona a modo de ``chequeo de sanidad'', como una red de seguridad
para incrementar la confianza de que nuestros cálculos están bien y no hemos cometido algún error en el camino. Algo similar
al sistema de tipos de un lenguaje de programación.

Otro uso más sofisticado consiste en analizar las dimensiones de las cantidades involucradas para deducir propiedades físicas.
Como pequeño ejemplo, supongamos que vamos a tratar de hallar una expresión para el tiempo $t$ que tarda en caer al suelo un objeto
soltado desde una altura $h$ en el planeta tierra, y supongamos además que sabemos (o quizá conjeturamos) que las únicas
cantidades físicas que pueden afectar el valor de $t$ son $h$, la masa $m$ del objeto, su volumen $V$ y la aceleración de la
gravedad $g$. Bajo estas hipótesis, notaremos que las dimensiones correspondientes a $h,m,V,g$ son $\longitud,\masa,\longitud^3,\frac{\longitud}{\tiempo^2}$.
Luego si suponemos además que la ecuación para $t$ tiene la siguiente pinta general\footnote{Tarea: Piense cuántas leyes físicas que conozca no tienen esa pinta}:

$t = C h^am^bV^cg^d$

Siendo $C,a,b,c,d$ constantes. Podemos cambiar a dimensiones para verificar la validez de la misma:

$\tiempo = \longitud^a\masa^b(\longitud^3)^c(\frac{\longitud}{\tiempo^2})^d$

Pero de aquí es inmediato que debe ser $b=0, d = -\frac{1}{2}$. De aquí resulta que en estas condiciones, el tiempo que tarda
en caer un cuerpo deberá ser independiente de la masa, y, fijados los demás parámetros, inversamente proporcional a la raíz de la aceleración
de la gravedad. Ambas cosas son ciertas, y las deduciremos por consideraciones dinámicas más adelante en el libro, pero notemos
que sin saber absolutamente nada de la naturaleza del sistema físico en cuestión, hemos logrado realizar afirmaciones no triviales
sobre su comportamiento basados en el análisis dimensional (También hemos necesitado ciertas asunciones sobre qué factores pueden afectar al fenómeno
bajo estudio).

Estas ideas pueden formalizarse a través del Teorema Buckingham $\Pi$, teorema central del análisis dimensional. Se invita
al lector interesado a investigar más en wikipedia, internet en general o por qué no, una biblioteca. Nosotros no daremos
mucha importancia al análisis dimensional en este sentido, limitándonos al uso para revisar la validez de nuestras expresiones
y asegurarnos de que las dimensiones son adecuadas.

Un uso intermedio del análisis dimensional consiste en utilizarlo de manera similar al ejemplo para conjeturar leyes plausibles, y luego
experimentar y plantear teorías y deducciones matemáticas que justifiquen las conjeturas. Es decir, se lo puede usar como una valiosa
ayuda a la intuición.

\subsection{La matemática con unidades}

\subsubsection{El análisis dimensional en la matemática}

La matemática usual, en la forma en que la ve y presenta típicamente un matemático, prescinde de las dimensiones por completo.
Esto no trae problemas formales ni de expresividad, ya que es esencialmente equivalente a trabajar con un sistema de unidades
básicas fijo todo el tiempo, y entonces siempre que se hagan cuentas válidas en el sentido del análisis dimensional no habrá
diferencia. Más aún, pueden permitirse cuentas inválidas según nuestro criterio usual usando un sistema de unidades natural (ver
más adelante).

Es de alguna manera un poco triste, a mi entender, que en la práctica usual algunos estudiantes de matemática
ignoren las cuestiones de dimensionalidad subyacentes en todas estas operaciones fundamentales. Mueve números para
todos lados, pero al no tener en cuenta que si derivamos una función de espacio en términos de tiempo, el resultado
tendrá dimensiones de velocidad, perdemos capacidad de razonar intuitivamente e interpretar geométricamente y físicamente
los conceptos y resultados del análisis.

Un primer ejemplo concreto de ello son los espacios vectoriales, particularmente los espacios vectoriales con producto interno.
Ya hemos mostrado como podemos extender la definición natural para pensar en términos de unidades, de manera que la norma de un
vector no siempre será un mero escalar adimensional: Si el vector es un desplazamiento espacial, su norma tendrá dimension de
longitud, es decir, tiene sentido decir que nos movemos $\SI{3}{\meter}$, pero no tiene sentido decir que nos movemos $28$.

\subsubsection{Derivadas}

La derivada afecta la dimensión al dividir por la dimensión del dominio. La definición de derivada es:

$$f'(x) = \lim_{h \rightarrow 0}{\frac{f(x+h) - f(x)}{h}}$$

Por lo tanto si $f : D_1 \rightarrow D_2$ es una función entre dos dimensiones, la expresión del límite tiene dimensión $\frac{D_2}{D_1}$.
De ahí que si derivamos una función que va de $\tiempo$ en $\longitud$, su derivada tiene dimensión de velocidad, es decir, $\longitud \tiempo^{-1}$

\subsubsection{Integrales}

Similarmente, la integral multiplica por la dimensión del dominio. 

Algo similar ocurre si no se visualiza la integral como el limite de una suma. Justamente, visto el $dx$ ahi multiplicando,
esta claro que aporta a la dimension del resultado y no puede ser tomado a la ligera.

Si $f : D_1 \rightarrow D_2$ como en el ejemplo anterior, $\int_{a}^{b}{f(x) dx}$ se puede ver como el límite de las sumas
de Riemann, $\sum_{i=1}^{n}{f(a + i \Delta x) \Delta x}$, siendo $\Delta x = \frac{b-a}{n}$. Esta visión de la integral como
el límite de una suma, es extremadamente importante en física. Muchos resultados son muy intuitivos y fáciles de entender
cuando se piensa a la integral como el límite de una suma de estas características, como por ejemplo el teorema de la divergencia
en varias variables.

Notar en nuestro ejemplo, que la expresión del límite tiene ahora unidades $D_1 D_2$. Esto es totalmente coherente además, con el
hecho de que la derivación y la integración son operaciones inversas.

\subsection{Sobre la arbitrariedad de la elección de dimensiones}

Acá contar sobre sistemas de unidades naturales y sarasa. O sea, ya dijimos y mostramos claramente que cuáles
dimensiones tomamos como fundamentales y cuales como derivadas es arbitrario (conviene enfatizarlo aquí igualmente).
Pero aquí la idea es ejemplificar que qué dimensiones existen, también es arbitrario. Las dimensiones no son más
que un objeto matemático que nos ha resultado extremadamente útil para razonar, pero que no es imprescindible para describir el
universo (y un poco podría servirle como defensa a los matemáticos que se cagan en las unidades en su trabajo, suponiendo
todo adimensional).

Empezamos con el ejemplo concreto de los grados, y mostramos como los grados pueden ser considerados una unidad para una dimensión
diferente en sí, la extensión de un ángulo. Así, ecuaciones como velocidad angular * radio para conseguir la velocidad lineal asociada,
ahora tendrían una constante física metida en el medio (probablemente, la constante que indica en ángulo correspondiente a una circunferencia
entera). Tendríamos una unidad extra, pero todo sería matemáticamente equivalente. Ahora bien, como nos dimos cuenta de que se puede
medir los ángulos de manera *natural* directamente haciendo arco / radio, es una dimensión que no usamos, y eso redunda en que desaparecen
constantes físicas. O sea que esas constantes físicas en realidad no son tales, no son parámetros del universo que uno podría mover
libremente a su antojo, sino que resultan de la elección de dimensiones nuestra. En una elección de dimensiones diferente, en una
elección *natural*, desaparecerían.

Bueno, después de eso mostrar el ejemplito de la velocidada de la luz, que te hace desaparecer tiempo / longitud (te las pegotea en una sola).

Y luego comentar que se puede hacer lo mismo en física hasta hacer desaparecer todas las dimensiones, cargándones varias constantes físicas
en el medio, como la constante de Planck y G. Ahora no hay dimensiones y todo se mide con numeritos adimensionales.

Comentar que esto genera ambigüedad más fácil que antes ("5", está hablando de un tiempo o una longitud?), además de que las unidades
naturales en general no resultan prácticas (unidades de planck, muy chiquitas :P).

\pagebreak

\section{El espaciotiempo}

\subsection{Estructura afín del espacio tiempo}

El espaciotiempo se puede ver de dos maneras: El espacio vectorial de saltos o desplazamientos espaciotemporales, o
el espacio afin de eventos (cuyo espacio vectorial de traslaciones asociado es justamente uno de los primeros).

\subsection{Comentario sobre relatividad general}

Esta noción nos servirá para física clásica y para relatividad especial, pero en relatividad general se va todo al carajo
porque el espacio tiempo ahora es curvo (y con una curvatura que depende de las masas que lo habitan), con lo cual será
una variedad diferencial que localmente se parece a un espaciotiempo de Minkowski. No vamos a tratar en este libro
relatividad general.

\subsection{Introducción a los observadores inerciales, y discusión sobre su importancia}

Vamos a comentar una vez explicado más o menos eso ahora que va a haber una noción importante que es la de "observador inercial".

Un observador inercial es un sistema de coordenadas afín del espacio tiempo, pero no uno cualquiera: es justamente uno
en el cual las leyes de la fisica toman su expresion más simple. La geometria del espacio tiempo medio que
vendra dada justamente por las transformaciones bajo las cuales las leyes de la fisica no varian (algo así
como que el espacio tiempo es más que un espacio vectorial y ya, tiene una ``geometría'' y cualquier automorfismo
del espaciotiempo que preserve esa geometría deja invariante las leyes de la física... Esa geometría puede ser
el producto interno o similares, o como en relatividad, una forma bilineal simétrica de signatura +++-).

Dadas las leyes de la física para un observador inercial, las transformaciones de coordenadas entre observadores inerciales
se deducen de las leyes de la física, ya que un observador inercial es justamente un sist. de coordenadas en el cual
las leyes de la física toman esa misma forma (o sea, es físicamente indistinguible de nuestro observador de partida).
Nosotros las deduciremos a partir de algunos axiomas básicos sobre las leyes de la física, y luego entonces lo que pediremos
será que todas las leyes sean invariantes respecto a los sistemas así definidos, ya que si no fuera así, tendremos que
modificar nuestras nociones de observadores inerciales y revisar los axiomas (por ejemplo, lo haremos al pasar de física
clásica a relatividad especial).

En realidad podríamos ignorar la geometría del espaciotiempo y trabajar con leyes más complicadas, que resulten
modelar solitas esa geometría, pero serían leyes chuecas inmanejables en lugar de leyes mucho más sencillas aprovechando
la geometría. Además
la noción de observador resulta ser fundamental en la práctica, ya se corresponde con un experimentador tomando
medidas en un sistema de referencia determinado, y por lo tanto es esencial poder transformar entre observadores.
Además las transformaciones son importantes porque permiten deducir propiedades físicas y hacer predicciones no triviales
relativas al tiempo y al espacio
sin conocer las leyes físicas concretas involucradas (lo que sabremos es que son invariantes entre sistemas inerciales, es decir,
al aplicar transformaciones de coordenadas entre sistemas inerciales).

Finalmente, es importante hablar de observadores desde temprano ya que si bien para ciertas cosas conviene tener una visión
global del espacio tiempo, para la mayor parte de la física planteamos todas las leyes y pensamos directamente en términos
de observadores (más precisamente, planteamos las leyes usando las coordenadas correspondientes a un observador),
ya que las leyes resultantes tienen una forma mucho más manejable e intuitiva. Por ejemplo el concepto de energía cinética
(o energía en general) no es absoluto y depende del observador. Así una partícula con masa puede tener en un instante dado
cualquier cantidad de energía cinética, dependiendo del observador en cuestión, de manera que ``la energía cinética de una
partícula'', en abstracto sin conocer el observador respecto al cual medimos, es algo vacío de significado. Sin embargo
la energía es un concepto clave, absolutamente esencial y conveniente en la física, y no podemos darnos el lujo de ignorarlo para
trabajar solo con conceptos absolutos (o al menos hacerlo no sería práctico ni útil ni elegante). Otro ejemplo de esto
es que en mecánica siempre estudiaremos la posición de un cuerpo como función del tiempo, algo natural y súper práctico.
Ahora bien, en términos absolutos si miramos una trayectoria en el espaciotiempo, no es claro a qué llamar ``tiempo'',
y con lo cual expresiones como la ecuación diferencial de un péndulo se vuelven un enjambre indescriptible y completamente
carente de intuición.

Por todo esto es que hablaremos de observadores inerciales y cambios de coordenadas entre ellos desde temprano, junto al
estudio del espaciotiempo mismo. De esta manera entonces, construiremos los observadores de manera particular
y distinta para la física clásica y para relatividad especial, ya que ambos enfoques describen espaciotiempos con geometrías
diferentes: pero en ambos casos los observadores se corresponderán en la práctica con la idea de tomar un sistema dado por un
cuerpo rígido moviéndose libre por el espaciotiempo, sin interactuar con nada (es decir, librado a su inercia y nada más).

\subsection{Observadores inerciales en física clásica}

Acá vamos a deducir las transformaciones de coordenadas entre observadores inerciales para el caso de la física clásica,
a partir de las hipótesis de que la distancia temporal y la distancia espacial (de eventos simultáneos) son valores
absolutos que no dependen del observador (o sea que dos observadores cualesquiera están de acuerdo sobre estos valores).

\pagebreak

\section{Mecánica clásica del punto}

AL CARAJO, PARA LOS ESPACIOS VECTORIALES, HABLAREMOS DE MULTIPLICARLOS POR UNA CANTIDAD SOLO CUANDO SEAN ESPACIOS
CON PRODUCTO INTERNO (O AL MENOS CON UNA FORMA BILINEAL ASOCIADA...)

De esta manera se puede hablar del espacio "soporte" digamo'. Cosas como el producto cruz y blah van a estar
definidas sobre espacios con el mismo soporte.




\end{document}
