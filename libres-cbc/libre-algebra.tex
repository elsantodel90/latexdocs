\documentclass{article}

% El archivo está codificado utf8.

\usepackage[utf8]{inputenc}


\usepackage[spanish]{babel}
\usepackage{amssymb}
\usepackage{amsmath}
\usepackage{amsthm}
\usepackage{graphicx}
\usepackage{xspace}

\setlength{\textwidth}{6.5in}
\setlength{\oddsidemargin}{0in}
\setlength{\textheight}{8.5in}
\setlength{\topmargin}{0.5in}
\setlength{\headheight}{0in}
\setlength{\headsep}{0in}

\def\lg{\mathop{\mathrm {lg}}\nolimits}

\title{Libre álgebra}
\author{}
\date{}

\makeindex

\newtheorem{teorema}{{\sc Teorema}}
\newtheorem{definicion}{{\sc Definición}}
\newtheorem{corolario}{{\sc Corolario}}
\newtheorem{lema}{{\sc Lema}}

%Ejemplos de macros
%\newcommand{\Left}{\textbf{Left}\xspace}
%\newcommand{\Right}{\textbf{Right}\xspace}
%\newcommand{\Jota}{\ensuremath{\mathcal{J}}\xspace}
%\newcommand{\Juego}[2]{\ensuremath{\left \{ #1 | #2 \right \}}\xspace}

\begin{document}

\maketitle

\section{Problema 1}

Considere la recta $\mathbb{L} : \lambda (-1,0,1) + (9,2,-13)$ y el punto $A = (1,2,-1)$

Hallar todos los pares de puntos $B,C \in \mathbb{L}$ tales que el triángulo $ABC$ sea rectángulo en $B$ e isósceles.

\subsection{Solucion}

Sean $P = (9,2,-13)$ y $v = (-1,0,1)$. Para ciertos $\lambda_1,\lambda_2$ tenemos:

$$B = P + \lambda_1 v$$

$$C = P + \lambda_2 v$$

La condicion de que $ABC$ sea rectangulo en $B$ se escribe como:

$$(A-B)\cdot(C-B) = 0$$

$$(A-B)\cdot((\lambda_2-\lambda_1) v) = 0$$

$$(\lambda_2-\lambda_1) (A-B)\cdot v = 0$$

Como para que sea triangulo $B \neq C$, tenemos $\lambda_1 \neq \lambda_2$ y entonces:

$$(A-B) \cdot v = 0$$

$$(A-P - \lambda_1 v) \cdot v = 0$$

$$(A-P) \cdot v - \lambda_1 v \cdot v = 0$$

$$\lambda_1 = \frac{(A-P) \cdot v}{v \cdot v}$$

Lo cual nos fija $B$. Por otra parte, la condicion de que sea isosceles implica que:

$$||A-B|| = ||C-B||$$

$$||A-B|| = ||(\lambda_2-\lambda_1) v||$$

$$||A-B|| = |\lambda_2-\lambda_1|\ ||v||$$

$$\frac{||A-B||}{||v||} = |\lambda_2-\lambda_1|$$

$$\lambda_2 = \lambda_1 \pm \frac{||A-B||}{||v||}$$

Lo cual nos fija los dos valores posibles de $C$. (Dejo las cuentitas a cargo del lector)

\section{Problema 2}

Sean en $\mathbb{R}^4$ los subespacios $\mathbb{S} = \{ x\in \mathbb{R}^4 | 2x_1 + x_3 - x_4 = 0 \}$ y
$\mathbb{T} = \left \langle (1,2,0,2), (1,a,b,0), (0,0,1,a) \right \rangle$. Hallar todos los pares de numeros
reales $(a,b)$ tales que $\dim(\mathbb{S} \cap \mathbb{T}) \neq 2$

\subsection{Solucion}

Es trivial verificar que $dim(\mathbb{T}) = 3$ (porque los tres generadores dados son LI) salvo que sean $a=2, b=-1$. En dicho caso, resulta

$$\mathbb{T} = \left \langle (1,2,0,2), (1,2,-1,0), (0,0,1,2) \right \rangle = \left \langle (1,2,0,2), (0,0,1,2) \right \rangle$$

Notemos que $(1,2,0,2) \in \mathbb{S}$, pero $(0,0,1,2) \notin \mathbb{S}$. Luego 

$$\alpha (1,2,0,2) + \beta (0,0,1,2) \in \mathbb{S} \Leftrightarrow \beta = 0$$

Con lo cual $\mathbb{S} \cap \mathbb{T} = \left \langle (1,2,0,2) \right \rangle$, luego $\dim(\mathbb{S} \cap \mathbb{T}) = 1 \neq 2$,
luego $a = 2, b = -1$ es un par posible.

Si $(a,b) \neq (2,-1)$, entonces $dim(\mathbb{T}) = 3$. Luego cada elemento $v \in \mathbb{T}$ se escribe de forma unica como:

$$v = \alpha (1,2,0,2) + \beta (1,a,b,0) + \gamma (0,0,1,a) $$

Si llamamos $f : \mathbb{R}^4 \rightarrow \mathbb{R}$ dada por $f(x_1,x_2,x_3,x_4) = 2x_1 + x_3 - x_4$, por definicion resulta
$\mathbb{S} = Nu(f)$. Como $(1,2,0,2) \in \mathbb{S}$ resulta que para $v \in Nu(f)$:

$$f(v) = \alpha f(1,2,0,2) + \beta f(1,a,b,0) + \gamma f(0,0,1,a) = \beta (2+b) + \gamma (1-a) = 0$$

Si $(a,b) = (1,-2)$, entonces $f(v) = 0 \forall v \in \mathbb{T}$, y por lo tanto $\mathbb{S} \cap \mathbb{T} = \mathbb{T}$ y su dimension es 3.
Luego $(a,b) = (1,-2)$ es otro par posible.

Estos son los unicos pares posibles, ya que en otro caso, de $\beta (2+b) + \gamma (1-a) = 0$ se deduce que $dim(\mathbb{S} \cap \mathbb{T}) = 2$.

\section{Problema 3}

Considere el polinomio $P(x) = x^4 - 6x^3 + ax^2 + bx + 27$. Hallar $a$ y $b$ de tal manera que la suma de dos de las raices de $P$
sea 4 y el producto de las otras dos sea 3.

\subsection{Solucion}

Sean las raices $x_1,x_2,x_3,x_4$, en algun orden. Podemos escribir la condicion del enunciado como:

$$x_1 + x_2 = 4$$

$$x_3 x_4 = 3$$

Es sabido que en un polinomio monico como $P$, $-(x_1 + x_2 + x_3 + x_4)$ es el coeficiente que sigue al coeficiente principal.
Luego en nuestro caso $x_1 + x_2 + x_3 + x_4 = 6 \Rightarrow x_3 + x_4 = 2$.

De manera similar, el termino independiente es $(-1)^n X$, con $X$ el producto de las raices y $n$ el grado del polinomio monico.
Luego en nuestro caso $x_1x_2x_3x_4 = 27 \Rightarrow x_1x_2 = 9$.

De todo esto resulta que $x_1,x_2$ son las raices de $X^2 - 4X + 9$, y $x_3,x_4$ son las raices de $X^2 -2X +3$.
Por lo tanto debe ser $P(X) = (X^2-4X+9)(X^2-2X+3)$. Con esto obtenemos $P(X)$, y por lo tanto podemos determinar los unicos
valores posibles de $a$ y $b$.

\section{Problema 3'}

Hallar todos los $z \in \mathbb{C}$ tales que $z^4|z|^2 + 16|z| = 16\sqrt{3}i|z|$

¿Cual es el minimo grado que puede tener un polinomio $P \in \mathbb{R}[x]$ tal que todas las soluciones de dicha ecuacion sean raices de $P$ ?

\subsection{Solucion}

(Recordatorio: Para $\alpha \in \mathbb{R}$, se tiene $e^{\alpha i} = \cos \alpha + i \sin \alpha$. Si no se entiende porque, se
puede tomar como una notacion, que resulta consistente con las propiedades de la exponencial y el producto y suma de complejos gracias
al teorema de Moivre)

Claramente $z=0$ cumple. Si $z\neq0$, podemos dividir por $|z|$ obteniendo:

$$z^4|z| = 32(-\frac{1}{2} + \frac{\sqrt{3}}{2} i)$$

Tomando modulo de ambos lados queda:

$$|z|^5 = 32 \Rightarrow |z| = 2$$

Reemplazando en la anterior:

$$z^4 = 16(-\frac{1}{2} + \frac{\sqrt{3}}{2} i) = 16e^{\frac{2 \pi}{3} i}$$

$$z = 2e^{\frac{(2 + 6k) \pi}{12} i}$$

Con $k \in \{ 0,1,2,3 \}$ (son las raices de la anterior porque las 4 cumplen). Luego hemos encontrado las 5 soluciones posibles.

El grado minimo de un polinomio real que contenga a estos 5 complejos como raiz es 9. En efecto, si un polinomio con coeficientes reales
tiene una raiz compleja $z$, el conjugado $\overline z$ tambien es raiz. Luego los 4 conjugados de los $z$ no nulos encontrados deben
ser raices tambien de un polinomio que contenga a todos esos $z$. Claramente el polinomio monico que tiene como raices exactamente
a esos 9 complejos es real (porque cada vez que hay una raiz compleja, esta su conjugada), con lo cual estamos.

\section{Problema 4}

Sean en $\mathbb{R}^4$ los subespacios $\mathbb{S} = \{ x \in \mathbb{R}^4 | x_1 + 2x_3 - x_4 = x_2 + x_3 - x_4 = 0 \}$
y $\mathbb{T} = \{ x \in \mathbb{R}^4 | x_1 + x_2 + x_3 = x_2 - x_3 + x_4 = 0 \}$. Definir, si es posible, una
t.l. $f : \mathbb{R}^4 \rightarrow \mathbb{R}^4$ que satisfaga simultaneamente $Nu f = \mathbb{S}$, $Im f = \mathbb{T}$ y
$(f \circ f)(1,0,0,1) = (f \circ f)(0,0,1,0) $

\subsection{Solucion}

Como la composicion de t.l. es t.l, resulta:

$$(f \circ f)(1,0,0,1) = (f \circ f)(0,0,1,0) \Leftrightarrow (f \circ f)(1,0,-1,1) = 0 $$

Luego debe ser $f(f(1,0,-1,1)) = 0 \Leftrightarrow f(1,0,-1,1) \in Nu f$. Ademas trivialmente $f(1,0,-1,1) \in Im f$, luego
$f(1,0,-1,1) \in Nu f \cap Im f = \mathbb{S} \cap \mathbb{T}$

Calculando $\mathbb{S} \cap \mathbb{T}$ (es buscar el nucleo del sistemita de 4 ecuaciones que sale de pegotear las ecuaciones
de ambos), podemos tomar un vector $v \in \mathbb{S} \cap \mathbb{T}$, y luego completar con un $w$ de modo que $\{v,w\}$ sea base de $\mathbb{T}$.

Si tomamos por otro lado una base $\{v_1,v_2\}$ de $\mathbb{S}$, y la extendemos con los vectores $(1,0,-1,1),v_4$ de modo que $\{v_1,v_2,(1,0,-1,1),v_4\}$
sea base de $\mathbb{R}^4$ (esto es posible porque $(1,0,-1,1)$ no esta en $\mathbb{S}$), si definimos $f$ tal que:

$$f(v_1) = f(v_2) = 0$$

$$f(1,0,-1,1) = v$$

$$f(v_4) = w$$

Es facil comprobar que esta $f$ satisface todo lo pedido. He demostrado que existe la $f$ y he dado un procedimiento mecanico para construirla,
usando operaciones estandar como extender conjuntos l.i. a bases, buscar nucleos, etc: Le dejo la construccion al lector (por otra parte,
esta bueno hacer construcciones de esta forma en lugar de mandarse de una a las cuentas porque uno puede visualizar que carajo esta haciendo).

\section{Problema 5}

Sea $B = \{v_1,v_2,v_3\}$ una base de $\mathbb{R}^3$ y sea $f : \mathbb{R}^3 \rightarrow \mathbb{R}^3$ la t.l. tal que

$$M_B(f) = \begin{pmatrix} 1 & 2 & 0 \\ -1 & 0 & 2 \\ 0 & 1 & 1 \end{pmatrix}$$

Hallar, si existe, una base $B'$ de $\mathbb{R}^3$ tal que

$$M_{B'}(f) = \begin{pmatrix} -1 & 1 & 2 \\ 0 & 1 & 0 \\ -1 & 1 & 2 \end{pmatrix}$$

\subsection{Solucion}

(Problema dificil para los conocimientos de algebra del CBC. Doy un sketch de solucion en la que pego un galerazo tremendo en un punto.)

Si dos matrices codifican la misma transformacion lineal en distintas bases, se dice que las matrices son \textit{semejantes}.
Esto equivale a que exista una matriz inversible $C$ tal que $A = CBC^{-1}$. $C$ resulta ser basicamente la matriz cambio de base
que pasa de una de las bases a la otra. La semejanza es una relacion de equivalencia, que quiere decir que es es reflexiva,
simetrica y transitiva (como la igualdad), o sea que de alguna manera particiona al mundo en bolsas de cosas que ``se parecen''
entre si.

Entonces, en el problema nos dan dos matrices y lo que nos piden basicamente es decidir si son semejantes, y si lo son, dar la $C$
(nos piden la base $B'$, pero basicamente la base $B'$ en terminos de la $B$ es un dato que se lee directamente de la matriz cambio
de base $C$).

El approach general para ver que no son semejantes, es encontrar alguna propiedad intrinseca de la transformacion lineal $f$, que sea
calculable a partir de la matriz en cualquier base (pero que la propiedad no depende de la base, que sea intrinseca a la $f$ en si),
y ver que en ambas matrices difiere. Ejemplos podrian ser el determinante ($(-1)^n$ veces el producto de todos los autovalores, con multiplicidad, luego
no depende de la base), la traza (suma de los elementos de la diagonal, que da la suma de los autovalores y por lo tanto no depende
de la base), polinomio caracteristico (la traza y el determinante son dos coeficientes de este), etc.

El approach general para ver que si son semejantes, es llevar cada una a alguna ``forma canonica'', o bien simplemente a cualquier matriz
semejante, pero la misma para las dos. Lo que digo es: si $A = C_1MC_1^{-1}$, y $B = C_2MC_2^{-1}$, entonces $A = C_1C_2^{-1}BC_2C_1^{-1} =
C_1C_2^{-1}B(C_1C_2^{-1})^{-1}$. Luego si hacemos un cambio de base a cada una que lleve a ambas a una misma matriz, componiendo esos
cambios de base adecuadamente obtenemos un cambio de base entre las dos matrices que queremos relacionar. Este es el approach que
voy a usar, porque estas matrices dadas son semejantes.

La ``forma canonica'' a usar puede ser cualquier cosa, pero hay una obvia: Una matriz diagonal. Si dos matrices son diagonalizables, es
claro que son semejantes si y solo si diagonalizan a matrices con los mismos autovalores, y si es asi, diagonalizando cada una podemos
hacer lo que queremos.

En este caso, si miramos las matrices, no son diagonalizables, pero cada una tiene un autovector de autovalor $1$ y uno de autovalor $0$.
Si metemos esto en las bases correspondientes, esas columnas tienen un solo elemento no nulo en la diagonal, que corresponde a esos autovalores
0 y 1. Como no son diagonalizables, no podemos pretender encontrarle otro autovector l.i. a ninguna de las dos, pero si encontramos por ejemplo
un vector $v$ tal que $Mv = v + w$, siendo $w$ el autovector de autovalor 1 elegido para la base, si completamos la base con $v$ ademas de
los dos autovectores, en la columna de $v$ la matriz tiene solo dos unos, uno en la fila de $v$ y otro en la de $w$. Luego si encontramos
un $v$ de estas caracteristicas en cada matriz, estamos hechos porque para cada una tenemos un cambio de base que lo lleva a una matriz con
la pinta esa que dije, y por lo tanto combinando bien esos cambio de base nos sale un cambio de base entre ambas y con eso estamos.

El $v$ ese se puede encontrar buscando la solucion al sistemita lineal $(M-id)v = w$, que tiene solucion para ambas matrices en nuestro caso.

Dejo todas las cuentas al lector. Es muy posible que no se entienda porque esta muy mal explicado :P

\section{Problema 5'}

Sean $B = \{ v_1,v_2,v_3 \}$ y $B' = \{v_2,v_2+v_3,v_1-v_2\}$ bases de $\mathbb{R}^3$. Consideramos la t.l. 
$f : \mathbb{R}^3 \rightarrow \mathbb{R}^3$
tal que:

$$M_{BB'}(f) = \begin{pmatrix} -11 & -5 & -7 \\ 5 & 2 & 3 \\ \alpha & 0 & 0 \end{pmatrix}$$

Hallar todos los $\alpha \in \mathbb{R}$ para los cuales existe una base $B*$ de $\mathbb{R}^3$ con

$$M_{B*}(f) = \begin{pmatrix} 0 & 1 & 0 \\ 1 & 0 & 0 \\ 0 & 0 & \alpha^2 \end{pmatrix}$$

\subsection{Solucion}

Este es mas facil que el anterior aunque no parezca :P

Notemos que en base a la primer matriz que nos dan se tiene:

$$f(v_1) = -11v_2 + 5(v_2 + v_3) + \alpha(v_1-v_2) = \alpha v_1 + (-6 -\alpha) v_2 + 5v_3$$

$$f(v_2) = -5v_2 + 2(v_2 + v_3) = -3v_2 + 2v_3$$

$$f(v_3) = -7v_2 + 3(v_2 + v_3) = -4v_2 + 3v_3$$

De aqui nos sale que

$$M_{B}(f) = \begin{pmatrix} \alpha & 0 & 0 \\ -6-\alpha & -3 & -4 \\ 5 & 2 & 3 \end{pmatrix}$$

Con lo cual podemos observar que el problema equivale a ver cuando esta matriz es semejante a la segunda matriz (semejanza definida
en el problema anterior).

El polinomio caracteristico de esta matriz es $(X-\alpha)(X^2-1)$. Como esta matriz debe ser semejante a la propuesta
$M_{B*}(f)$, deben tener el mismo caracteristico. Pero el caracteristico de esta segunda matriz es
$(X^2-1)(X-\alpha^2)$. De aqui resulta que debe ser $\alpha = \alpha^2$ o sea $\alpha = 0,1$.

Si $\alpha = 0$ , ambas son diagonalizables (porque tienen 3 autovalores distintos), y con los mismos autovalores, luego son semejantes
a la misma matriz diagonal, y por lo tanto son semejantes entre si.


Si $\alpha = 1$, Podemos notar que la segunda matriz es diagonalizable, pero la primera no (si hice bien las cuentas, revisen). 
Por lo tanto en ese caso no son semejantes.

Luego el unico valor posible es $\alpha = 0$.

\end{document}
