\documentclass{article}

% El archivo está codificado utf8.

\usepackage[utf8]{inputenc}


\usepackage[spanish]{babel}
\usepackage{amssymb}
\usepackage{amsmath}
\usepackage{amsthm}
\usepackage{graphicx}
\usepackage{xspace}

\setlength{\textwidth}{6.5in}
\setlength{\oddsidemargin}{0in}
\setlength{\textheight}{8.5in}
\setlength{\topmargin}{0.5in}
\setlength{\headheight}{0in}
\setlength{\headsep}{0in}

\def\lg{\mathop{\mathrm {lg}}\nolimits}

\title{Sobre algunos algoritmos de camino mínimo en toda su generalidad}
\author{Agustín Gutiérrez}
\date{}

\makeindex

\newtheorem{teorema}{{\sc Teorema}}
\newtheorem{definicion}{{\sc Definición}}
\newtheorem{corolario}{{\sc Corolario}}
\newtheorem{lema}{{\sc Lema}}

%Macros mágicas.
%\newcommand{\Left}{\textbf{Left}\xspace}
%\newcommand{\Right}{\textbf{Right}\xspace}
%\newcommand{\Jota}{\ensuremath{\mathcal{J}}\xspace}
%\newcommand{\Juego}[2]{\ensuremath{\left \{ #1 | #2 \right \}}\xspace}
\begin{document}

\maketitle

\pagebreak

\tableofcontents

\pagebreak

\section{Introducción}

La intención de este artículo es explicar los algoritmos fundamentales para camino mínimo en grafos.
Sin embargo, el enfoque pretende fundamentalmente dejar explícitadas todas las hipótesis que necesita
cada uno de los algoritmos (e idealmente, solo ellas, es decir, no forzar hipótesis innecesarias), de
tal forma que sea posible aplicarlos en toda su generalidad.

Esta visión resulta útil al encarar problemas.

\section{Definiciones fundamentales}

Sea $G$ un grafo dirigido y sea $E$ su conjunto de arcos. Sea $D$ un conjunto de distancias totalmente ordenado por $\leq$, y sea $0 \in D$ un
elemento destacado particular de $D$. Además, pediremos que $D$ contenga un máximo, que llamaremos $+\infty$
(notar que esta restricción no es grave, pues siempre se puede agregar tal elemento extendiendo el orden de $D$), y un
mínimo que llamaremos $-\infty$.
Pediremos además que $D$ sea completo (En el sentido de que se pueda tomar ínfimo de un conjunto no vacío).
Sea $F : D \times E \rightarrow D$ una \textbf{función acumuladora}. Dado un camino $P = (e_1,\cdots,e_n)$ en $G$
entre $v_0$ y $v_n$, que pasa por $v_1, \cdots, v_{n-1}$, donde los $e_i \in E$ son los arcos del camino,
definiremos la \textbf{longitud} de $P$ como:

$$l(P) = F(F(F(F(F(0,e_1),e_2),e_3),\cdots),e_{n-1}),e_n)$$

Diremos que un camino $P$ entre $v_0$ y $v_n$ es \textbf{mínimo} si no existe otro camino entre $v_0$ y $v_n$ de menor
longitud. Llamaremos \textbf{distancia} entre $v_0$ y $v_n$, y la notaremos $d(v_0,v_n)$, al ínfimo de las longitudes
de caminos entre $v_0$ y $v_n$. Si no existe ningún camino entre $v_0$ y $v_n$, diremos que la distancia entre ellos es $+\infty$.

Notar que con esta definición, un camino es mínimo si y solo sí su longitud es igual a la distancia entre el punto
de origen y el punto de llegada, y que la distancia entre todo par de vértices siempre está definida y no es
necesariamente simétrica. Notar además que no necesariamente existe un camino mínimo, y en caso de existir, no
necesariamente este es único.

El problema del que nos ocuparemos es, entonces, computar distancias y caminos mínimos entre pares de puntos en un grafo. Cada
algoritmo le impondrá ciertas restricciones a la función acumuladora para su correcto funcionamiento.

\section{Algoritmo de Dijkstra}

El algoritmo de Dijkstra supone:

1) No negatividad : `` Caminar más nunca es bueno ''

$$F(d,e) \geq d \ \ \ \  \forall d \in D, e \in E$$

2) Principio del óptimo : `` Siempre conviene llegar antes a un nodo intermedio del camino ''

$$d_1 \leq d_2 \Rightarrow F(d_1,e) \leq F(d_2,e) \ \ \ \  \forall d_1,d_2 \in D, e \in E$$

Cualquier función que cumpla estas dos condiciones puede ser utilizada con el algoritmo de Dijkstra, que
computa las distancias y caminos mínimos desde un nodo origen $v_0$ a todos los demás.

\begin{proof}

Veamos que para el conjunto de vértices que Dijkstra marca como ``finalizados'', la distancia al origen es correcta.
Inicialmente, esto vale, pues se marca solamente $v_0$ con distancia $0$. Por la no negatividad, la
longitud de cualquier camino de $v_0$ a $v_0$ ha de ser $l(P) \geq 0$, luego la distancia es correcta y el camino trivial (que
no contiene ejes) resulta un camino mínimo.

En un paso dado del algoritmo, se toma el nodo $v$ no finalizado con la menor marca de distancia, y se lo finaliza. Veamos
que esto es correcto. Es claro que existe un camino $P$ entre $v_0$ y $v$ que realiza dicha distancia, y por lo tanto basta con probar
que cualquier otro camino es igual o más largo que $P$ para mostra que $P$ es camino mínimo. Para esto, sea $P'$ un camino
entre $v_0$ y $v$, y consideremos el primer nodo $u$ no finalizado que recorre $P'$. Si $u=v$, entonces es
inmediato por la no negatividad que $l(P') \geq l(P)$. Si $u \neq v$, por la elección de $v$, y como $u$ no está
finalizado, debe ser la marca de distancia de $u$ mayor o igual a la de $v$. Del principio del óptimo y la no
negatividad resulta $l(P') \geq l(P)$.

\end{proof}

Notar por lo tanto, que como dijkstra guarda durante su ejecución un camino que realiza la distancia, resulta
que para cualquier función que cumpla las condiciones 1) y 2), existen caminos mínimos entre todo par de nodos
(siempre y cuando exista al menos un camino). Otra forma de ver esto es que la no negatividad asegura que de
cualquier camino no simple se puede extraer un camino simple igual o más corto, luego se puede restringir la
atención a caminos simples, que son finitos, y por lo tanto el camino mínimo estará bien definido. Notar que
para este último argumento solo hemos necesitado 1), luego la no negatividad alcanza para garantizar la
existencia de caminos mínimos (siempre y cuando exista al menos un camino).

\subsection{Algoritmo basado en relajar nodos por orden creciente de distancias / programación dinámica.}

Notar que lo único relevante para el correcto funcionamiento del algoritmo de Dijkstra es el hecho de que va
finalizando los nodos en orden creciente de distancias al origen. Por lo tanto, cualquier algoritmo que de alguna
manera se garantice finalizar los nodos en orden creciente de distancias al origen (BFS, BFS con cola de dos puntas,
BFS multicolas, etc etc etc) será correcto bajo las mismas hipótesis que el algoritmo de Dijkstra.

De hecho, es fácil pensar el algoritmo de Dijkstra como un algoritmo de programación dinámica si uno ya sabe el
orden de distancias al origen de los nodos: la distancia en cada nodo será el mínimo sobre los ``padres'' del
nodo de la distancia del padre más la longitud del arco del padre al nodo. En este caso, los ``padres'' son los
nodos a menor distancia que son antecesores directos en el grafo. Se cumple principio del óptimo y
superposición de subproblemas, características distintivas de la programación dinámica.

\subsection{Algoritmo de programación dinámica en DAGs.}

Con exactamente la misma idea del apartado anterior, en un DAG es posible dar un algoritmo de programación dinámica
para calcular caminos mínimos. En este caso, es posible prescindir de la condición 1), aunque claramente necesitamos
que valga el principio del óptimo. Este uso es típico por ejemplo en el problema del tablerito de $n \times m$
que hay que ir de una esquina a la otra yendo solo para abajo y para la derecha.

\section{Algoritmo de Bellman - Ford}

El algoritmo de Bellman-Ford asume el principio del óptimo. Es una programación dinámica, que llena una tablita
que mapea pares (cantidad de aristas en el camino, $nodo$) \texttt{->} distancia del origen a $nodo$. Es claro que con
la tablita para caminos con $k$ aristas se computa fácilmente la tablita para $k+1$ aristas. Notar que estos
valores estan perfectamente definidos por estar fijada la cantidad de aristas, habiendo así finitos caminos
para cada caso.

De esta manera, el algoritmo computa en $O(mL)$ las longitudes de los caminos mas cortos del origen a todos los
demas nodos, cuya longitud sea $l \in {0,\cdots, L}$. Por lo tanto, si se puede acotar la longitud del camino
buscado, este algoritmo es una opcion posible.

\section{Algoritmo de Floyd}

El algoritmo de Floyd asume una versión fuerte del principio del óptimo, y una estructura particular sobre la
función acumuladora. Específicamente, asume que $D$ es un monoide con elemento neutro $0$ bajo la operación
dada por $\cdot$, que existe una $g : E \rightarrow D$ tal que $F(d,e) = d \cdot g(e)$ para $d \in D, e \in E$, y
que el monoide es compatible con el orden de $D$, es decir, 
$d_1 \leq d_2 \Rightarrow d_1 \cdot d \leq d_2 \cdot d$ y $d \cdot d_1 \leq d \cdot d_2$.

El hecho de que este monoide sea compatible con el orden de $D$ implica una versión fuerte del principio del óptimo
en el siguiente sentido: La primer condición $d_1 \cdot d \leq d_2 \cdot d$ implica $F(d_1,e) \leq F(d_2,e)$, la
versión simple del ya mencionado principio del óptimo (`` Siempre conviene llegar antes a un nodo intermedio del camino '').

La segunda condición, $d \cdot d_1 \leq d \cdot d_2$, dice algo nuevo: Implica que además, sin importar cómo sea
que uno llega a un nodo dado, la relación entre qué subcaminos de dicho nodo al nodo destino producirán un camino
total más corto es siempre la misma. En particular, si uno tiene un camino mínimo de $a$ a $b$, y un camino mínimo
de $b$ a $c$, y los concatena, obtiene un camino mínimo de $a$ a $c$.

La expresión vulgarmente conocida de este algoritmo sería (para vértices numerados de 1 a $n$):

\begin{verbatim}

Para cada k <- [1..n]
Para cada i <- [1..n]
Para cada j <- [1..n]
           dist[i][j] = min(dist[i][j], dist[i][k] * dist[k][j] )

\end{verbatim}

Donde con el \texttt{*} denotamos la operación $\cdot$ ya introducida.

No obstante, esta no sería la versión ``completa'' del algoritmo en toda su totalidad, que mostraremos a continuación.

El algoritmo es un algoritmo de programación dinámica que computa una tabla \texttt{dp[k][i][j]}, que indica el
ínfimo de las longitudes de caminos entre $i$ y $j$, \textbf{que utilizan como nodos intermedios únicamente nodos
del conjunto \texttt{[1..k]}}.

Para $k=0$, no puede haber nodos intermedios, por lo tanto \texttt{dp[k][i][j]}
con $i \neq j$ indica el mínimo $F(0,e) = g(e)$, siendo $e$ arista entre $i$ y $j$, $\infty$ si no existe tal arista.
\texttt{dp[k][i][i]} indica el mínimo entre $0$ y $F(0,e) = g(e)$ con $e$ autoeje de $i$. Estos números pueden computarse
directamente, y forman el caso base de la programación dinámica.

Veamos como computar la tabla para $k$ a partir de la tabla para $k-1$. Llamamos por comodidad:

$$d(i,j) = dp[k-1][i][j]$$
$$d'(i,j) = dp[k][i][j]$$

Por definición, $d'(i,j)$ es el ínfimo de las longitudes de caminos de $i$ a $j$ que utilizan como nodos
intermedios nodos del conjunto \texttt{[1..k]}. Dado que en $d(i,j)$ ya tenemos considerados los caminos de $i$ a $j$ que
utilizan nodos intermedios del conjunto \texttt{[1..k-1]}, solo nos falta considerar el uso del nodo $k$.

Es fácil ver entonces que:

$$d'(i,j) = \min(d(i,j), I)$$
$$I = \inf \left \{d(i,k) \cdot d(k,k)^i  \cdot d(k,j) | i \in \mathbb{Z}_{\geq 0} \right \}$$

Por supuesto, en una implementacion no es posible calcular $I$ aplicando directamente su definición pues involucra
un conjunto infinito, sino que se debe utilizar alguna expresión analítica usando la forma particular del problema.
En muchos casos lo que ocurre es que tal ínfimo se reduce simplemente a su elemento con $i=0$, es decir,
$I = d(i,k) \cdot d(k,j)$, dando lugar de esta manera al algoritmo descrito anteriormente.

Notar que el algoritmo de Floyd cambiando $\min$ por la suma y $\inf$ por una sumatoria infinita, 
tomando como $0 \in D$ al $1$, $D = \mathbb{Z}_{\geq 0} U \{ +\infty \}$,
usando como $\cdot$ el producto usual, y tomando $g(e) = 1$ para todo $e \in E$, resulta un algoritmo
para calcular la cantidad de caminos entre cada par de nodos (posiblemente infinitos). Notar que en
este caso, $I$ se calcula fácilmente pues si alguno de $d(i,k)$,$d(k,k)$,$d(k,j)$ es cero, la suma
se calcula muy fácilmente evaluando finitos términos, y sino será $I= +\infty$.

Ocurrira lo mismo con Bellman - Ford y el algoritmo para exponenciacion de matrices.

CHE ESTOY ASUMIENDO ESTO QUE HAY QUE VER SI VALE O QUE HIPOTESIS PONER:

$$\inf (S \cdot T) = \inf \{ s \cdot t | s \in S, t \in T \} = \inf (S) \cdot \inf (T)$$

\section{Algoritmo de Dantzig}

\section{Algoritmo basado en exponenciación de matrices}

\end{document}
