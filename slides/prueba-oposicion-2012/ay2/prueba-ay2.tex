\documentclass{beamer}

\mode<presentation>
{
  \usetheme{Warsaw}
  \useoutertheme{infolines}
  \usecolortheme[RGB={125,173,51}]{structure}
  %\usetheme[height=7mm]{Rochester}
  % or ...

  \setbeamercovered{transparent}
  % or whatever (possibly just delete it)
}

\usepackage{multicol}
\usepackage{verbatim} 
\usepackage{listings}
\usepackage{tikz}
\usetikzlibrary{arrows}
\usetikzlibrary{shapes}
\tikzstyle{block}=[draw opacity=0.7,line width=1.4cm]

\lstloadlanguages{C++}
\lstnewenvironment{code}
	{%\lstset{	numbers=none, frame=lines, basicstyle=\small\ttfamily, }%
	 \csname lst@SetFirstLabel\endcsname}
	{\csname lst@SaveFirstLabel\endcsname}
\lstset{% general command to set parameter(s)
	language=C++, basicstyle=\footnotesize\ttfamily, keywordstyle=\slshape,
	emph=[1]{tipo,usa}, emphstyle={[1]\sffamily\bfseries},
	morekeywords={tint,forn,forsn},
	basewidth={0.47em,0.40em},
	columns=fixed, fontadjust, resetmargins, xrightmargin=5pt, xleftmargin=15pt,
	flexiblecolumns=false, tabsize=2, breaklines,	breakatwhitespace=false, extendedchars=true,
	numbers=left, numberstyle=\tiny, stepnumber=1, numbersep=9pt,
	frame=l, framesep=3pt,
}

\usepackage[spanish]{babel}
% or whatever

\usepackage[latin1]{inputenc}
% or whatever

\usepackage{times}
\usepackage[T1]{fontenc}
% Or whatever. Note that the encoding and the font should match. If T1
% does not look nice, try deleting the line with the fontenc.


\title{Divide \& Conquer}  % (optional, use only with long paper titles)


\author[Agust�n Guti�rrez] % (optional, use only with lots of authors)
{~Agust�n Santiago Guti�rrez}
% - Give the names in the same order as the appear in the paper.
% - Use the \inst{?} command only if the authors have different
%   affiliation.
\institute[UBA] % (optional, but mostly needed)
{
  %\inst{1}
  Facultad de Ciencias Exactas y Naturales\\
  Universidad de Buenos Aires
}
\date[Octubre 2012] % (optional, should be abbreviation of conference name)
{Octubre 2012}

% Ac� se puede insertar el logo de la UBA
% \pgfdeclareimage[height=0.5cm]{university-logo}{university-logo-filename}
% \logo{\pgfuseimage{university-logo}}



% Delete this, if you do not want the table of contents to pop up at
% the beginning of each subsection:
\AtBeginSubsection[]
{
  \begin{frame}{Contenidos}
  \footnotesize
  \begin{multicols}{2} 
    \tableofcontents[currentsection, currentsubsection]
  \end{multicols}
  \end{frame}
}

\DeclareMathOperator*{\mimin}{min}
\DeclareMathOperator*{\mimax}{max}

% If you wish to uncover everything in a step-wise fashion, uncomment
% the following command: 

%\beamerdefaultoverlayspecification{<+->}


\begin{document}
\pgfdeclarelayer{background}
\pgfsetlayers{background,main}
\begin{frame}
  \titlepage
\end{frame}


\begin{frame}{�En qu� consiste un algoritmo de divide \& conquer?}

  \begin{itemize}
   \item Muchos algoritmos de utilidad son recursivos: para resolver un problema, se utilizan las soluciones a subproblemas fuertemente relacionados.
   \pause
   \item Estos algoritmos suelen ajustarse a un esquema de divide and conquer: Se divide el problema en varios subproblemas 
   que luego se resuelven y se combinan las soluciones obtenidas para resolver el original.
  \end{itemize}
  
\end{frame}

\begin{frame}{El esquema general}
   \begin{itemize}
   \item Los algoritmos de divide and conquer se organizan en 3 pasos que responden al siguiente esquema general:
   \pause
   \item \textbf{Divide:} Se divide el problema en subproblemas m�s peque�os (m�s f�ciles), similares al problema original.
   \item \textbf{Conquer:} Se resuelven cada uno de los subproblemas: Esto se hace t�picamente de manera recursiva, invocando el mismo
     algoritmo de divide and conquer, pero al llegar a instancias suficientemente peque�as se utiliza alg�n otro m�todo como caso base
     de la recursi�n.
   \item \textbf{Combine:} Se combinan las soluciones de los subproblemas para obtener una soluci�n al problema original
  \end{itemize}
\end{frame}

\begin{frame}{Referencias}
   \begin{itemize}
   \item Introduction to Algorithms, 2nd Edition. MIT Press. Thomas H. Cormen, Charles E. Leiserson, Ronald L. Rivest, Clifford Stein
   P�gina 28: 2.3.1 The divide-and-conquer approach

   
  \end{itemize}
\end{frame}

\begin{frame}{El problema de armar fixtures}
	Ejercicio 3.4, pr�ctica 3 de Algoritmos y Estructuras de Datos 3:

\ 

	Hay que organizar un torneo que involucre a $n$ competidores. Cada competidor debe jugar exactamente una vez contra
	cada uno de sus oponentes. Adem�s cada competidor debe jugar un partido por d�a con la sola posible excepci�n de un d�a
	en el cual no juegue.
	\pause
   \begin{itemize}
   \item[a] Si $n$ es potencia de 2, dar un algoritmo que use la t�cnica de dividir y conquistar para armar el fixture para que el torneo
   termine en $n-1$ d�as.
   \item[b] Si $n > 1$ no es potencia de 2, dar un algoritmo para armar el fixture para que el torneo termine en $n-1$ d�as si $n$ es par,
   o en $n$ si $n$ es impar.
  \end{itemize}
\end{frame}

\begin{frame}{El problema de armar fixtures (cont)}
	Supondremos que tenemos disponible una operaci�n:
	
	\texttt{jugar(fecha, jugador1, jugador2)}
	
	que reporta que se debe jugar un partido entre esos
    jugadores en esa fecha (Podr�a consistir en loguear el evento por ejemplo, o en agregarlo a una lista de partidos para ser procesada luego).
    
    Numeraremos los jugadores con enteros consecutivos desde 0 hasta $n-1$, e igualmente numeraremos las fechas con enteros desde 0.
   
\end{frame}


\begin{frame}[fragile]{Soluci�n (parte a)}
	Utilizando divide and conquer:
\begin{lstlisting}
reportarPartidos(listaJugadores):
    n <- cantidad de jugadores
    // Si hay un solo jugador, no hace falta jugar nada.
    SI n > 1: 
        j1 <- lista de los primeros n/2 jugadores de listaJugadores
        j2 <- lista de los ultimos n/2 jugadores de listaJugadores
        // Armamos las primeras fechas del fixture recursivamente
        reportarPartidos(j1)
        reportarPartidos(j2)
        // Y armamos las �ltimas n/2 fechas para obtener el fixture final
        PARA i <- 0 HASTA n/2 - 1:
            PARA j <- 0 HASTA n/2 - 1:
                jugar(n/2 - 1 + i, j1[j], j2[(i+j)%(n/2)])
\end{lstlisting}
\end{frame}

\begin{frame}[fragile]{Soluci�n (parte b)}
	�C�mo se puede adaptar la idea de la parte a para resolver la parte b?
	\pause
	\invisible<1>
	{
		\begin{itemize}
			\item Para resolver los casos impares, podemos agregar un jugador ``centinela'', de manera que ahora una fecha donde un jugador no juega
			  se interpreta como una fecha donde juega contra el centinela. El problema equivale ahora a armar un fixture con los $n+1$ jugadores
			  que quedan, y podemos reducir el problema al caso par.
			\item En el divide and conquer del caso par, ahora puede pasar que $\frac{n}{2}$ resulte ser impar (aunque $n$ sea par), y por
			  lo tanto los subproblemas contendr�n partidos con jugadores libres. Si emparejamos en cada fecha a los jugadores libres de
			  cada torneo, podemos proceder similar a como hac�amos en la parte a y armar un fixture en $n-1$ fechas.
		\end{itemize}
	}
\end{frame}


\end{document}
