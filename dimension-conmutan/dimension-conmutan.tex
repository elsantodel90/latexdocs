\documentclass{article}

% El archivo está codificado utf8.

\usepackage[utf8]{inputenc}


\usepackage[spanish]{babel}
\usepackage{amssymb}
\usepackage{amsmath}
\usepackage{amsthm}
\usepackage{graphicx}
\usepackage{xspace}

\setlength{\textwidth}{6.5in}
\setlength{\oddsidemargin}{0in}
\setlength{\textheight}{8.5in}
\setlength{\topmargin}{0.5in}
\setlength{\headheight}{0in}
\setlength{\headsep}{0in}

\def\lg{\mathop{\mathrm {lg}}\nolimits}

\title{Espacio de matrices que conmutan con una dada}
\author{}
\date{}

\makeindex

\newtheorem{teorema}{{\sc Teorema}}
\newtheorem{definicion}{{\sc Definición}}
\newtheorem{corolario}{{\sc Corolario}}
\newtheorem{lema}{{\sc Lema}}

\begin{document}

\maketitle

Sea $V$ un $K$-espacio vectorial de dimension finita. Sea $f$ un endomorfismo en $V$. Es sabido que el conjunto
de los endomorfismos $g$ tales que $fg = gf$ forman un subespacio del espacio de endomorfismos de $V$, que llamaremos $E_f$. Vamos a
caracterizarlo, suponiendo que $m_f$, el minimal de $f$, se factoriza linealmente en $K$. En particular, vamos a dar una
expresion para su dimension en base a los bloques de la forma de jordan de $f$.

Si $m_f = PQ$, con $P,Q$ coprimos, entonces $V = Nu(P(f)) \oplus Nu(Q(f))$, ambos $f$-invariantes. Si $g \in E_f$, es inmediato comprobar que
$Nu(P(f))$ y $Nu(Q(f))$ son $g$-invariantes. Luego podemos considerar la restriccion de $g$ a estos subespacios, y $g$ conmutara cuando cada
una de estas restricciones lo haga con las correspondientes restricciones de $f$.

Luego tenemos que estudiar ahora el caso en que $m_f = (X-\lambda)^t$. Si llamamos a partir de ahora $f$ a lo que seria $f - \lambda id$,
es trivial que no cambia el espacio $E_f$, porque todo conmuta con la identidad. Luego falta ver solamente quienes conmutan con una $f$
nilpotente, es decir, tal que $m_f = X^t$.

Consideremos una base de jordan de esta $f$, dada por:

$$v_1, f(v_1), \cdots, f^{r_1 - 1}(v_1)$$

$$\cdots$$

$$v_k, f(v_k), \cdots, f^{r_k - 1}(v_k)$$

Y tal que $f^{r_i}(v_i) = 0$.

Supongamos que $w_i = g(v_i)$. Notemos que para que $g$ conmute con $f$, debe ser para cualquier $t$:

$$g(f^t(v_i)) = f^t(w_i)$$

En particular, $0 = g(0) = g(f^{r_i}(v_i)) = f^{r_i}(w_i)$.

Luego, vemos que todos los $g \in E_f$ se pueden obtener fijando un $w_i$ tal que $w_i \in Nu(f^{r_i})$, y
definiendo la $g$ sobre el resto de la base de Jordan segun $g(f^t(v_i)) = f^t(w_i)$ . La pregunta
que falta es si todos los $g$ que se obtienen por este mecanismo son un $g \in E_f$, pero esto se
comprueba mecanicamente ya que por la definicion dada, $g$ y $f$ conmutan en los vectores de la base de Jordan,
luego conmutan en general.

Juntando todo esto hemos dado una caracterizacion de los elementos de $E_f$. De la misma se observa que su
dimension ha de ser:

$$\sum_{\lambda}{\sum_{i=1}^{\infty}{\gamma(f,\lambda,i) \dim \left ( Nu((f-\lambda I)^i) \right )}}$$

Donde notamos $\gamma(f,\lambda,i)$ a la cantidad de bloques de Jordan de $f$ de autovalor $\lambda$ de tamaño $i$.

Un caso especial de interes se da cuando $m_f = \chi_f$. En este caso, el espacio de los endomorfismos que son
un polinomio de $f$ tiene dimension $n$ (su dimension en general es el grado de $m_f$) y como puede observarse,
en este caso el espacio de las $g$ que conmutan con $f$ tambien tiene dimension $n$. Como los $P(f)$ conmutan
con $f$, resulta que en este caso el espacio de las $g=P(f)$ coincide con $E_f$.

\end{document}
