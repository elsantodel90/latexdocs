\documentclass{article}

% El archivo está codificado utf8.

\usepackage[utf8]{inputenc}


\usepackage[spanish]{babel}
\usepackage{amssymb}
\usepackage{amsmath}
\usepackage{amsthm}
\usepackage{graphicx}
\usepackage{xspace}

\setlength{\textwidth}{6.5in}
\setlength{\oddsidemargin}{0in}
\setlength{\textheight}{8.5in}
\setlength{\topmargin}{0.5in}
\setlength{\headheight}{0in}
\setlength{\headsep}{0in}

\def\lg{\mathop{\mathrm {lg}}\nolimits}
\newcommand\numberthis{\refstepcounter{equation}\tag{\theequation}}

\title{Ejercicio de suma: repaso de inducción}
\author{}
\date{}

\makeindex

\begin{document}

\maketitle

\section{Enunciado}

$$ \mbox{Demostrar que } \forall n \in \mathbb{N} \mbox{ vale que } \sum_{i=1}^{n}{i \cdot 2^i} = (n-1) 2^{n+1} + 2 $$

\section{Demostración por inducción}

Sea $P(n) \equiv \sum_{i=1}^{n}{i \cdot 2^i} = (n-1) 2^{n+1} + 2$. Se nos pide demostrar $(\forall n \in \mathbb{N}) P(n)$. Lo haremos por inducción en $n$.


\subsection{Caso base}
Debemos probar $P(1)$.

Por definición, $P(1) \equiv \sum_{i=1}^{1}{i \cdot 2^i} = (1-1) 2^{1+1} + 2 \Leftrightarrow 1 \cdot 2^1 = 2$, que es trivialmente cierto. Luego queda
probado $P(1)$, completando el caso base.

\subsection{Paso inductivo}
Debemos probar $(\forall n \in \mathbb{N}) (P(n) \Rightarrow P(n+1))$.

Supongamos entonces que $n \in \mathbb{N}$ es un natural cualquiera tal que se cumple $P(n)$. Con esta hipótesis inductiva, tenemos que demostrar $P(n+1)$, que por definición es:

\begin{align*}
\sum_{i=1}^{n+1}{i \cdot 2^i} & = ((n+1)-1) 2^{(n+1)+1} + 2 \Leftrightarrow \\
\Leftrightarrow (n+1) 2^{n+1} + \sum_{i=1}^{n}{i \cdot 2^i} & = n \cdot 2^{n+2} + 2 \numberthis \label{eq:1}
\end{align*}

Pero recordemos que estamos suponiendo que vale la hipótesis inductiva, es decir, \\ $P(n) \equiv \sum_{i=1}^{n}{i \cdot 2^i} = (n-1) 2^{n+1} + 2$.
Reemplazando de esta forma la sumatoria en el lado izquierdo de \ref{eq:1}, lo que tenemos que demostrar queda:

\begin{align*}
(n+1) 2^{n+1} + (n-1) 2^{n+1} + 2 & =n \cdot 2^{n+2} + 2 \Leftrightarrow \\
\Leftrightarrow \left( (n+1) + (n-1) \right) 2^{n+1} & =n \cdot 2^{n+2} \Leftrightarrow \\
\Leftrightarrow 2n \cdot 2^{n+1} & =n \cdot 2^{n+2} \\
\end{align*}

Lo cual es válido ya que $2 \cdot 2^{n+1} = 2^{n+2}$. Con esto hemos probado lo que queríamos, es decir, $P(n+1)$, partiendo de la suposición de que vale $P(n)$.
Esto completa el paso inductivo, y con eso la demostración. \qed

\section{Bonus: Demostración alternativa}

Una demostración alternativa ``directa'', sin inducción sería por ejemplo la siguiente:

$$\sum_{i=1}^{n}{i \cdot 2^i} = \sum_{i=1}^{n}\sum_{j=1}^{i}{2^i} = \sum_{j=1}^{n}\sum_{i=j}^{n}{2^i}$$

Usando la fórmula de la suma de la serie geométrica\footnote{$\sum_{i=0}^{n-1}{a_0 \cdot r^i} = a_0 \cdot \frac{r^n-1}{r-1}$. En nuestro caso será $r=2$, $a_0$ es el primer término de la suma, y $n$ la cantidad total de términos.} queda:

$$\sum_{j=1}^{n}{2^j(2^{n-j+1} - 1)} = \sum_{j=1}^{n}{\left(2^{n+1} - 2^j\right)} = \sum_{j=1}^{n}{2^{n+1}} - \sum_{j=1}^{n}{2^j} = n \cdot 2^{n+1} - (2^{n+1} - 2) = (n-1) 2^{n+1} + 2$$

Donde hemos usado nuevamente la fórmula de la suma de la serie geométrica para eliminar la sumatoria. \qed

\end{document}
