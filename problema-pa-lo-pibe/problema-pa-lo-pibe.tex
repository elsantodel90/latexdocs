\documentclass{article}

% El archivo está codificado utf8.

\usepackage[utf8]{inputenc}


\usepackage[spanish]{babel}
\usepackage{amssymb}
\usepackage{amsmath}
\usepackage{amsthm}
\usepackage{graphicx}
\usepackage{xspace}

\setlength{\textwidth}{6.5in}
\setlength{\oddsidemargin}{0in}
\setlength{\textheight}{8.5in}
\setlength{\topmargin}{0.5in}
\setlength{\headheight}{0in}
\setlength{\headsep}{0in}

\def\lg{\mathop{\mathrm {lg}}\nolimits}

\title{Brevísima lista de problema' pa' lo' pibe'}
\author{Agustín Gutiérrez}
\date{}

\makeindex

\newtheorem{problema}{{\sc Problema}}
\newtheorem{definicion}{{\sc Definición}}
\newtheorem{corolario}{{\sc Corolario}}
\newtheorem{lema}{{\sc Lema}}

%Macros mágicas.
\newcommand{\Left}{\textbf{Left}\xspace}
\newcommand{\Right}{\textbf{Right}\xspace}
\newcommand{\Jota}{\ensuremath{\mathcal{J}}\xspace}
\newcommand{\Juego}[2]{\ensuremath{\left \{ #1 | #2 \right \}}\xspace}
\begin{document}

\maketitle

\begin{problema}[Teorema de la galería de arte]
Sea $P$ un polígono simple en el plano de $n$ lados. Un conjunto de puntos $S \subseteq P$ se dice que vigila $P$ si para cualquier
$p \in P$ se tiene que existe un $q \in S$ tal que $\overline{pq} \subseteq P$.

Demostrar que existe un $S$ que vigila a $P$ y tal que $|S| \leq \left \lfloor \frac{n}{3} \right \rfloor$

Aclaraciones: $P$ se entiende como el conjunto de puntos del polígono, tanto el interior como la frontera. Un polígono simple
en el plano es un polígono sin agujeros tal que sus lados no se intersecan, salvo lados consecutivos en sus extremos.
\end{problema}

\begin{problema}
Una hoja de papel rectangular se divide en $n$ regiones, todas de igual área. Otra hoja de papel de idénticas dimensiones
se divide también en $n$ regiones de igual área, aunque no necesariamente de la misma forma que la anterior.

Demostrar que es posible superponer una hoja sobre la otra, y pinchar $n$ alfileres a través de ellas, de manera que
cada alfiler atraviese exactamente un país en cada hoja, y cada país sea atravesado por exactamente un alfiler.
\end{problema}

\begin{problema}[Teorema de Fáry]
Sea $G$ un grafo simple planar. Demostrar que es posible dibujar $G$ en el plano de forma que todas las aristas se representen
por segmentos de recta.

Comentario: Es un problema abierto si lo anterior es posible pidiendo además que las longitudes de las aristas sean enteras. 
Está demostrado que sí es posible para el caso particular de grafos cúbicos.
\end{problema}

\begin{problema}
Demostrar que todo polígono simple en el plano admite una triangulación. Una triangulación de un polígono simple en el plano
es una partición del polígono en triángulos con vértices en vértices del polígono.
\end{problema}

\begin{problema}[Google CodeJam 2010]
Carolina y Ariel juegan al siguiente juego:

En el pizarrón se encuentran escritos los números $A$ y $B$, enteros positivos. Empieza Carolina y juegan una vez cada uno.
En su turno, un jugador debe elegir un $k$ entero positivo, y cambiar $A$ por $A - kB$, o bien cambiar $B$ por $B - kA$.
El primer jugador que escribe un número negativo o cero pierde.

Decidir para cada $A$ y $B$ quien tiene estrategia ganadora.
\end{problema}

\begin{problema}[``El algoritmo RSA funciona'']
Para cada $n,d$ enteros positivos, definimos $f_n^d : \mathbb{Z}_n \rightarrow \mathbb{Z}_n$ por $f_n^d(x) \equiv x^d (mod\ n)$
Dado un número $n$ entero positivo, decimos que $n$ es piola si vale la siguiente propiedad:

Para todo $d$ entero positivo, $d$ es coprimo con $\phi(n) \Leftrightarrow f_n^d$ es biyectiva.

Hallar todos los números piolas.
\end{problema}

\begin{problema}[Google CodeJam Africa 2010]
$N$ personas participan de la Code Jam Africa 2010, en la cual se les presentan $K$ problemas para resolver.
El problema $1$ fue resuelto por $S_1$ participantes, el $2$ por $S_2$, y en general el $i$ por $S_i$ con $1 \leq i \leq K$.

Demostrar que:

1) Si cada persona resolvió al menos $C$ problemas, $0 \leq C \leq K$, entonces para cada $I \subseteq \{1,2,\cdots,K\}$, con $|I| = t$ se tiene que:

$$N (K-t) + \sum_{i \in I}{S_i} \geq N C$$

2) Dado un $C$ tal que $0 \leq C \leq K$, si para todo $I \subseteq \{1,2,\cdots,K\}$ se verifica la desigualdad de 1), 
entonces es posible que cada persona haya resuelto al menos $C$ problemas 
(en el sentido de que existe una asignación de cuáles problemas resolvió cada persona que verifica todo lo enunciado y en la que esto ocurre).
\end{problema}

\end{document}
