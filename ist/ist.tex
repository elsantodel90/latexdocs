\documentclass[12pt]{article}

\usepackage[utf8]{inputenc}

\usepackage[spanish]{babel}
\usepackage{amssymb}
\usepackage{amsmath}
\usepackage{amsthm}
\usepackage{graphicx}
\usepackage{xspace}

\setlength{\textwidth}{6.5in}
\setlength{\oddsidemargin}{0in}
\setlength{\textheight}{8.5in}
\setlength{\topmargin}{0.5in}
\setlength{\headheight}{0in}
\setlength{\headsep}{0in}

\def\lg{\mathop{\mathrm {lg}}\nolimits}

\newtheorem{teorema}{{\sc Teorema}}
\newtheorem{definicion}{{\sc Definición}}
\newtheorem{corolario}{{\sc Corolario}}
\newtheorem{lema}{{\sc Lema}}

\title{Axiomas de Teorías de conjuntos}
\author{Agust\'\i n Guti\'errez}
\date{}

\begin{document}

\maketitle

\section{Preliminar}

En todas las teorias de conjuntos estandar que veremos, el lenguaje de la logica consta, ademas del igualador, de un unico relator
binario que denota la pertenencia entre conjuntos (clases en NBG), y notaremos con $\in$.

Notar que primero describimos las teorías de conjuntos ``básicas'', es decir, los axiomas que afirman qué conjuntos existen y las relaciones
entre ellos. A todas estas teorías básicas, se agregan luego axiomas especiales que afirman otros hechos importantes, como infinitud, elección, partes, etc.


\section{Zermelo (Z)}

La teoría de conjuntos de Zermelo (Z) fue la primera formalización razonable de la matemática (creada especialmente para evitar la paradoja
de Rusell presente en la teoría original de Frege, lo cual logró restringiendo el poder del axioma de especificación).

El universo de objetos sobre el que habla la teoria son los conjuntos. Los axiomas dados por Zermelo fueron:

1) $(\forall UV) ((\forall X) (X \in U \Leftrightarrow X \in V) \Rightarrow U = V)$ \ \ \ \ \ \ extensionalidad

2) $(\forall XY) (\exists Z) (\forall U) (U \in Z \Leftrightarrow U = X \vee U = Y)$ \ \ \ \ \ par

3) $(\exists Y) (\forall X) (X \notin Y)$ \ \ \ \ \  vacio

4) $(\forall X) (\exists Y) (\forall U) (U \in Y \Leftrightarrow (\exists V) (U \in V \wedge V \in X))$ \ \ \ \ \ \ union

5) Para toda formula $\phi(x)$ (tal vez con mas variables libres) la formula siguiente es axioma

$$(\forall A) (\exists B) (\forall X) (X \in B \Leftrightarrow X \in A \wedge \phi(X)) \mbox{\ \ \ \ \ \ \ \ especificacion}$$

\section{Zermelo-Fraenkel (ZF)}

La teoría de conjuntos de Zermelo-Fraenkel (ZF), surge cuando Fraenkel reemplaza el esquema de especificacion, por el mas fuerte esquema de
reemplazo:

5) Para toda formula $\phi(X,Y)$ (quiza con mas variables libres), vale el siguiente esquema axiomatico de reemplazo:

$$(\forall XYZ) ((\phi(X,Y) \wedge \phi(X,Z) \Rightarrow Y = Z) \Rightarrow (\forall A) (\exists B) (\forall Y) (Y \in B \Leftrightarrow (\exists X \in A) (\phi(X,Y))) )$$

El esquema de reemplazo es necesario para probar la existencia de ciertos conjuntos cociente , ciertos ordinales, probar ciertos resultados
abstractos de teoria de conjuntos, y otras yerbas, pero el grueso de la matematica tradicional puede llevarse a cabo con solo el axioma
de especificacion.

\section{von Neumann - Bernays - Gödel (NBG) y \\ Morse-Kelley (MK)}

La teoría de conjuntos de von Neumann - Bernays - Gödel (NBG) predica sobre las \textit{clases} de conjuntos.
La idea es que para cualquier propiedad bien definida exista la clase de los conjuntos con esa propiedad, por ejemplo,
la clase de todos los conjuntos, la clase de los conjuntos que no se pertenecen a si mismos, etc. El chiste es que una
clase no tiene por que ser un conjunto, con lo cual los axiomas en general persiguen dos objetivos principales: Asegurar
que cualquier propiedad define una clase, y asegurar que ciertas clases particulares de interés, son conjuntos. La
relación con ZF es bastante evidente (ZF habla exactamente de lo que en NBG son las clases que son conjuntos).

\subsection{Notación previa}

Se define $cto \ X$, ``$X$ es un conjunto'', como $\exists Y X \in Y$. Así, definimos los conjuntos como las clases que
pertenecen a alguna clase (o sea que las clases son algunas colecciones de conjuntos particulares, y sobre ellas habla
la teoría NBG).

Notaremos siempre a las clases con mayusculas y a los conjuntos con minusculas. Es decir, valen las siguientes abreviaciones

$$\forall x \alpha \equiv \forall X (cto\ X \Rightarrow \alpha)$$

$$\exists x \alpha \equiv \exists X (cto\ X \wedge \alpha)$$

Se definen las tuplas como:

$$\{Y_1, \cdots, Y_n \} \equiv Z | \forall u (u \in Z \Leftrightarrow u = Y_1 \vee \cdots \vee u = Y_n )$$

$$(Y_1) \equiv Y_1$$

$$(Y_1, Y_2) \equiv \{ \{ Y_1 \}, \{Y_1,Y_2 \} \}$$

$$(Y_1, \cdots ,Y_n) \equiv ((Y_1, \cdots , Y_{n-1}), Y_n)$$

Tambien definimos la abreviacion $Un \ X$, ``representar una funcion'', como:

$$Un \ X \equiv \forall uvw( (u,v) \in X \wedge (u,w) \in X \Rightarrow v = w)$$

\subsection{Axiomas}

\subsubsection{Axiomatizacion finita de NBG}

01) $(\forall XY)((\forall u)(u \in X \Leftrightarrow u \in Y) \Rightarrow X = Y)$ \ \ \ \ extensionalidad

02) $(\forall XY) (\exists Z) (\forall u)(u \in Z \Leftrightarrow u \in X \wedge u \in Y)$ \ \ \ \ \ interseccion

03) $(\forall X) (\exists Y) (\forall u) (u \in Y \Leftrightarrow u \notin X)$ \ \ \ \ \ \ \ \ \ \ \ \ \ \ \ \ complemento

04) $(\forall uv) (\exists y) (\forall x) (x \in y \Leftrightarrow x = u \vee x = v)$ \ \ \ \ \ \ \ \ \ \ par

05) $(\exists A) (\forall xy) ((x,y) \in A \Leftrightarrow x \in y)$ \ \ \ \ \ \ \ \ \ \ \ \ \ \ \ \ \ \ pertenencia

06) $(\forall A) (\exists B) (\forall x) (x \in B \Leftrightarrow (\exists y)( (x,y) \in A))$ \ \ \ \ \ \ \ dominio

07) $(\forall A) (\exists B) (\forall xy) ((x,y) \in B \Leftrightarrow x \in A)$ \ \ \ \ \ \ \ \ \ \ producto cartesiano

08) $(\forall A) (\exists B) (\forall xy) ((x,y) \in B \Leftrightarrow (y,x) \in A)$ \ \ \ \ \ \ \ \ relacion inversa

09) $(\forall A) (\exists B) (\forall xyz) ((x,y,z) \in B \Leftrightarrow (y,z,x) \in A)$ \ \ \ \ permutacion

10) $(\forall A) (\exists B) (\forall xyz) ((x,y,z) \in B \Leftrightarrow (x,z,y) \in A)$ \ \ \ \ permutacion

11) $(\exists x) (\forall y) (y \notin x)$ \ \ \ \ conjunto vacio

12) $(\forall x) (\exists y) (\forall uv) (u \in v \wedge v \in x \Rightarrow u \in y)$  \ \ \ \ \ union

13) $(\forall xA) (Un \ A \Rightarrow (\exists y) (\forall u) (u \in y \Leftrightarrow (\exists v \in x)( (v,u) \in A)))$ \ \ \ \ reemplazo

\subsubsection{Axiomatizacion con esquema axiomatico}

La anterior axiomatizacion es finita, lo cual tiene la propiedad interesante de mostrar que NBG (y en particular la matematica toda) puede
formalizarse con una cantidad finita de axiomas. No obstante esto, resulta engorrosa por tener que postular explicitamente la existencia
de muchos conjuntitos y clases muy especificos (como por ejemplo los axiomas de permutacion).

Una formula se llama primitiva cuando no cuantifica sobre clases, es decir, todas las cuantificaciones que aparecen son sobre conjuntos.
Puede probarse que con los axiomas anteriores, si $\phi$ es una formula primitiva, el siguiente es un teorema en NBG:

$$(\exists Y)(\forall x) (x \in Y \Leftrightarrow \phi)$$

Es decir, para cualquier propiedad que no cuantifique sobre clases, existe una clase cuyos elementos son exactamente los conjuntos que cumplen
la propiedad. Si tomamos esto como un esquema axiomatico, podemos obtener muchos de los axiomas anteriores a partir del mismo. Esto nos
permite enunciar la siguiente axiomatizacion equivalente de NBG (Que resulta mucho mas simple y usual, aunque tiene infinitos axiomas):

1) $(\forall XY)((\forall u)(u \in X \Leftrightarrow u \in Y) \Rightarrow X = Y)$ \ \ \ \ extensionalidad

2) Para toda formula primitiva $\phi$ (donde $Y$ no esta libre)

$$ (\exists Y)(\forall x) (x \in Y \Leftrightarrow \phi) \mbox{\ \ \ \ \ formacion de clases}$$ 

3) $(\forall uv) (\exists y) (\forall x) (x \in y \Leftrightarrow x = u \vee x = v)$ \ \ \ \ \ \ \ \ \ \ par

4) $(\exists x) (\forall y) (y \notin x)$ \ \ \ \ \ \ \ \ \  conjunto vacio

5) $(\forall x) (\exists y) (\forall uv) (u \in v \wedge v \in x \Rightarrow u \in y)$  \ \ \ \ \ union

6) $(\forall xA) (Un \ A \Rightarrow (\exists y) (\forall u) (u \in y \Leftrightarrow (\exists v \in x)( (v,u) \in A)))$ \ \ \ \ reemplazo

\subsubsection{Teoría de conjuntos de Morse Kelley (MK)}

La teoria de conjuntos de Morse Kelley (MK), se obtiene usando los 6 axiomas anteriores, pero en el esquema axiomatico de formacion de clases,
se elimina la restriccion a formulas primitivas (permitiendo usarla sobre cualquier formula donde $Y$ no este libre). La teoria resultante no
es equivalente a NBG si son consistentes.

\section{Axioma de infinitud}

Postula la existencia de un conjunto infinito. Hay dos versiones clasicas (equivalentes dados los otros axiomas)

1) Existe una funcion $f : x \rightarrow x $ inyectiva y no sobreyectiva. (definicion de infinitud de Dedekind)

2) Existe un conjunto inductivo (un conjunto que contiene a $\emptyset$, y que si contiene a $X$, contiene a $X \cup \{X \}$)

\section{Axioma de partes}

$$(\forall x)(\exists y) (\forall u) (u \subseteq y \Rightarrow u \in y)$$

Notar que todas las letras son minusculas intencionalmente: en NBG, este axioma habla de conjuntos.

\section{Axioma de regularidad}

$$(\forall x) (x \neq \emptyset \Rightarrow (\exists y \in x) (y \cap x = \emptyset))$$

Prohibe la existencia de conjuntos monstruos, pero por lo pronto el axioma este no es necesario para probar absolutamente nada importante.

Agregando infinitud, partes y regularidad a los axiomas basicos, obtenemos $ZF$ o $NBG$ sin axioma de eleccion.

\section{Axioma de elección}


$$(\forall x)(\exists f) (f \mbox{ es una funcion } \wedge \mathcal{D}f = x \wedge (\forall u \in x)(u \neq \emptyset \Rightarrow f(u) \in u))$$

O sea que para todo $x$ existe una \textit{funcion de eleccion} sobre $x$.

Dado que el axioma de eleccion no parece ni mas ni menos obvio que por ejemplo el de reemplazo, cabe preguntarse por que este es tan famoso y
genera tanto revuelo mientras que el de reemplazo no. 

La respuesta es que lo llamativo del axioma de eleccion no es para nada su obviedad o no obviedad, nociones que no tienen
sentido si no tenemos una nocion bien definida y entendida de conjunto (que no la tenemos), sino el caracter no constructivo del mismo:
Mientras que el axioma de reemplazo nos permite construir una funcion cuando tenemos un criterio bien definido para elegir un elemento
$f(x)$ para cada $x$, el axioma de eleccion nos dice que existe una funcion de eleccion, aunque no tenemos ni idea de cual puede haber sido
el elemento que elige de cada conjunto.

Al agregar a ZF el axioma de eleccion obtenemos ZFC. Al agregarlo a NBG obtenemos NBG con el axioma de eleccion (no estoy enterado de que haya
una sigla habitual para distinguir NBG con o sin axioma de eleccion sin aclararlo especialmente, aunque creo que NBG a secas se suele entender
como incluyendo el axioma de eleccion).

\section{Teoría de conjuntos no estándar (IST)}

La teoría de conjuntos no estándar que presentamos es la de Nelson (Hay otras, como la de Hrbacek, cuya relación con la de Nelson es análoga
a la relación de NBG con ZFC).

La teor\'\i a de conjuntos no est\'andar, llamada \textbf{Internal Set Theory}, utiliza el mismo lenguaje de l\'ogica que las teor\'\i as de conjuntos habituales,
salvo que se agrega un nuevo relator unario \textit{est\'andar}, notado como $st$.
As\'\i , $st\ x$ se lee \textit{$x$ es est\'andar}.


Una f\'ormula l\'ogica que \textbf{NO} utiliza el relator $st$ se denomina una f\'ormula interna, est\'andar o cl\'asica.
Naturalmente, una f\'ormula l\'ogica que utilice el citado relator se denomina externa, no est\'andar o no cl\'asica.
As\'\i , las f\'ormulas l\'ogicas de la teor\'\i a de conjuntos ZFC habitual son f\'ormulas internas.


Los axiomas de IST son los axiomas de ZFC, junto con tres axiomas adicionales (esquemas axiom\'aticos en realidad,
para ser precisos) que se describen a continuaci\'on.
No obstante, cabe destacar antes un detalle. En ZFC, algunos axiomas son esquemas axiom\'aticos, es decir,
un conjunto infinito de axiomas que se obtienen reemplazando en una expresi\'on general del axioma una cierta
f\'ormula l\'ogica. IST acepta los axiomas de ZFC, y en el caso de los esquemas axiom\'aticos, esto quiere decir
que IST toma los infinitos axiomas correspondientes que tiene ZFC. Pero en ZFC, no existe el relator $st$, y
por lo tanto la f\'ormula reemplazada en un esquema axiom\'atico debe ser necesariamente una f\'ormula interna
(simplemente porque en ZFC no existen las f\'ormulas externas, al no existir relator $st$). Por lo tanto,
las f\'ormulas l\'ogicas obtenidas al instanciar dichos esquemas axiom\'aticos con f\'ormulas externas \textbf{NO}
no son axiomas de ZFC, y por lo tanto tampoco lo son de IST. As\'\i , por ejemplo, no se puede aplicar el
axioma de especificaci\'on para construir un conjunto a partir de una f\'ormula externa.


Dicho esto, a continuaci\'on se muestran los axiomas de la teor\'\i a de conjuntos no est\'andar, habitualmente llamada IST (Internal Set Theory).
Notar que las iniciales de los nombres de los axiomas tambien son IST.


Aclaraciones de notaci\'on:


$(\forall^{st}x)(\phi)$ es una forma abreviada de escribir $(\forall x)(st \ x \Rightarrow \phi)$


$(\exists^{st}x)(\phi)$ es una forma abreviada de escribir $(\exists x)(st \ x \wedge \phi)$

Cuando decimos que una f\'ormula tiene ciertas variables libres, se entiende que de tener variables libres,
estas han de ser un subconjunto de las mencionadas, y no que la f\'ormula debe tener a todas esas variables
como variables libres presentes en la misma necesariamente.

Esquema axiom\'atico de Idealizaci\'on (\textbf{I}dealization):

Si $\phi$ es una f\'ormula interna, con variables libres $w_1, \cdots , w_n,f$ y $g$, la siguiente f\'ormula
constituye un axioma de IST:
$$(\forall w_1, \cdots , w_n)\left [ (\forall^{st} F \mbox{ finito})(\exists g)(\forall f \in F)(\phi) \Leftrightarrow 
(\exists g)(\forall^{st} f)(\phi) \right ]$$

Esquema axiom\'atico de Estandarizaci\'on (\textbf{S}tandarization):

Si $\phi$ es una f\'ormula (externa o interna), con variables libres $w_1, \cdots , w_n,x$ y $A$, la siguiente f\'ormula
constituye un axioma de IST:
$$(\forall w_1, \cdots , w_n)(\forall^{st} A)(\exists^{st} B)(\forall^{st} x)(x \in B \Leftrightarrow x \in A \wedge \phi)$$

Esquema axiom\'atico de Transferencia (\textbf{T}ransfer):

Si $\phi$ es una f\'ormula interna, con variables libres $w_1, \cdots , w_n$ y $x$, la siguiente f\'ormula
constituye un axioma de IST:
$$(\forall^{st} w_1, \cdots , w_n) \left [ (\forall^{st} x)(\phi) \Rightarrow (\forall x)(\phi) \right ]$$

Destacamos la interesante propiedad de que IST es una extensión conservativa de ZFC. Esto quiere decir,
que cualquier teorema de IST cuyo enunciado no use el relator $st$, es un teorema de ZFC (el recíproco es obvio
porque es una extensión). En particular, ZFC es consistente si y solo si IST lo es.

Esto se demuestra de forma constructiva, es decir se conoce un algoritmo concreto que dada una demostración de un enunciado
de ZFC usando IST, nos construye otra demostración del enunciado pero dentro de ZFC.

Notar que esto quiere decir que las demostraciones usando teoría de conjuntos no estándar (en particular, análisis no estándar
, números infinitos, infinitésimos, etc), son lógicamente incuestionables. 
A un matemático más ``clásico'' podrían llegar a parecerle ``feas'' o ``cuestionables'', pero no puede negarlas sin negar
al tradicional ZFC. Si prefiere pensarlo así, puede pensar que usar IST es simplemente un ``efecto especial'', un simple truquito
indirecto para construir finalmente una demostración ``de verdad''. Piense lo que quiera, funcionar funciona.

En particular, hay muchísimos casos en que las demostraciones en IST son (para muchos) más simples. Ejemplos clásicos se podrían dar
con resultados como Bolzano o Weirstrass (Weirstrass se puede demostrar de forma particularmente simple). Algunos también consideran
las definiciones mucho más intuitivas (Por ejemplo se puede definir la integral de una función estándar como la parte estándar de una
suma de Riemann donde los intervalos tienen longitud infinitesimal).

Como ejemplo solo para ver la ``pinta'' de una demostración no estándar, en un caso de ejemplo donde creemos que simplifica la demostración,
dejamos una demostración de un teorema de Sierpinski, con un par de aclaraciones entre paréntesis que no se incluirían para un lector habituado
a la teoría por ser triviales y conocidas (Antes de leer la demostración con teoría de conjuntos no estándar,
piensen y piensen y lleguen a una demostración clásica, como para comparar el estilo más claramente. Obviamente
acá no se pretende que el lector entienda la demostración, cosa imposible si no está familiarizado con la teoría,
simplemente nos parece interesante que aprecie la ``forma'' o ``pinta'' de la misma y la compare con su demostración clásica).

\begin{teorema}
    Si $a_1,\cdots,a_n,b$ son números reales positivos, la ecuación
     $$\frac{a_1}{x_1} + \cdots + \frac{a_n}{x_n} = b$$
     tiene a lo sumo un número finito de soluciones naturales.
    
\end{teorema}
\begin{proof}
   Podemos suponer sin pérdida de generalidad que los $n, a_1,\cdots,a_n,b$ son todos estándar (el axioma de transferencia garantiza automáticamente
   que si esto vale cuando esos son todos estándar, entonces vale siempre).
   Supongamos ahora que existen infinitas soluciones y lleguemos a un absurdo. 
   
   Tenemos que el conjunto de las soluciones es infinito y por lo tanto tiene un elemento no estándar.
      Si tomamos una solución no estándar $(x_1, \cdots, x_n)$, algunos $x_i$ serán no estándar (``infinitos'', en el sentido de que son
      más grandes que cualquier natural estándar, que son los que intuitivamente conocemos como ``naturales'', pero que no son todos los
      naturales de IST). Llamemos $A$ al conjunto de los $i$ tal que $x_i$ es estándar y $B$ al resto de los $i$ ($B$ es no vacío).
      
       Ahora bien,
      la suma $X = \sum_{i \in B} \frac{a_i}{x_i} \neq 0$ es un infinitésimo, luego $X$ es no estándar, pero por
      ser $(x_1,\cdots,x_n)$ solución de la ecuación, $X = b - \sum_{i \in A}\frac{a_i}{x_i}$, 
      luego $X$ debe ser estándar por ser suma, resta y cociente de números estándar. Absurdo. 
\end{proof}

Tenemos que destacar por otra parte, que para razonar de manera rigurosa, con seguridad y sin peligro de caer en contradicciones,
hace falta al trabajar en IST un mejor entendimiento de la lógica subyacente que con ZFC.

En otras palabras, un matemático que nada
sabe de lógica pero que trabaja en el framework clásico de ZFC es difícil que se mande una cagada no permitida por ZFC como
agarrarse el conjunto de todos los conjuntos o el conjunto de todos los cardinales o el conjunto de todos los conjuntos que no se pertenecen
a si mismos, mientras que en IST es mucho más fácil que si alguien no entiende bien la lógica de IST, se tiente a agarrarse cosas como
el conjunto de todos los naturales estándar, que no existe en la teoría.

\end{document}
